\section{Some Preliminaries}

\bx{
    \ea{
        \item Prove that $\sqrt{3}$ is irrational. Does a similar argument work to show $\sqrt{6}$ is rational?
        \item Where does the proof break down if we try to prove $\sqrt{4}$ is irrational?
    }
    \sol{
    \ea{
        \item 

        \begin{proof}
            \AFSOC that $\sqrt{3}$ is rational, so $\exists m, n \in \mathbb{Z}$, such that 
            \begin{equation*}
                \sqrt{3} = \frac{m}{n},
            \end{equation*}
            where $\frac{m}{n}$ is in the lowest reduced terms. By squaring both sides, we obtain $3 = (\frac{m}{n})^2 \implies 3n^2 = m^2$. Now, we know that $m^2$ is a multiple of 3 and thus $m$ must also be a multiple of 3. We can then write $m=3k$, deriving 
            \begin{align*}
                (\sqrt3)^2 &= \pa{\frac{3k}{n}}^2 \\
                3n^2&=9k^2 \\
                n^2 &= 3k^2
            \end{align*}

            Similar to above, we can conclude that n is a multiple of 3. However this is a contradiction since $m,n$ are both multiples of 3 but we assumed that $\frac{m}{n}$ was in its lowest reduced term. Thus we conclude that $\sqrt3$ is irrational.
        \end{proof}
        The same proof for $\sqrt3$ works for $\sqrt6$ as well.

            \item We cannot conclude that $\sqrt{4} = \frac{m}{n}$ imply that $m$ is a multiple of $4$, as we have 
        \begin{equation*}
            4n^2 = m^2 \implies 2n=m,
        \end{equation*}
        preventing us from reaching our contradiction that $m/n$ is not in its lowest terms.
    }
    }
}

\bx{
    Show that there is no rational number $r$ satisfying $2^r = 3$.
    \sol{
        \begin{proof}
            If $r=0$, then $2^r=1\neq3$. Suppose $r=p/q$ to get $2^p = 3^q$, which is not possible as $2$ and $3$ share no common factors. Hence $r$ is not rational.
        \end{proof}
    }
}

\bx{
    Decide which of the following represent true statements about the nature of sets. For any that are false, provide a specific example where the statement in question does not hold.
    \ea{
        \item If $A_1\supseteq A_2\subseteq A_3\subseteq A_4 \dots$ are all sets containing an infinite number of elements, then the intersections $\bigcap^\infty_{n=1}A_n$ is infinite as well.
        \item If $A_1\supseteq A_2\subseteq A_3\subseteq A_4 \dots$ are all finite, nonempty sets of real numbers, then the intersection $\bigcap^\infty_{n=1}A_n$ is finite and non-empty.
        \item $A\cap(B\cup C) = (A \cap B) \cup C$.
        \item $A \cap (B \cap C) = (A \cap B) \cap C$.
        \item $A \cap (B \cup C) = (A \cap B) \cup (A \cap C)$.
    }
    \sol{
        \ea{
            \item False. Consider $A_n = \{n, n+1, n+2, \dots\}$, then $\bigcap^\infty_{n=1}A_n = \emptyset$.
            \item True. Since all $A_n$ are nonempty, $\exists n \in \mathbb{N}$ such that $A_n = \{x\}$ for some real $x$. Hence $\bigcap^\infty_{n=1}A_n \subseteq \{x\}$ which is empty. Since $A_1$ is finite, $\bigcap^\infty_{n=1}A_n \subseteq \{x\} \subset A_1$ is finite.
            \item False. If $A=\emptyset$, then $\emptyset = C$
            \item True. Intersection is associative as evident that both LHS and RHS implies the $x \in A, B, C$
            \item True. Drawing a Venn Diagram illustrates this.
        }
    }
}

\bx{
    Produce an infinite collection of sets $A_1, A_2, A_3, \dots$ with the property that every $A_i$ has an infinite number of elements, $A_i \cap A_j = \emptyset$ for all $i \neq j$, and $\bigcup^\infty_{i=1}A_i = \mathbb{N}$.

    \sol{
        Consider arranging the elements of $\mathbb{N}$ in a square as such.
        \[
        \begin{matrix}
            1 & 3 & 6 & 10 & 15 & \dots \\
            2 & 5 & 9 & 14 & \dots & \\
            4 & 8 & 13 & \dots & & \\
            7 & 12 & \dots & & & \\
            11 & \dots & & & & \\
            \vdots & & & & & \\
        \end{matrix}
        \]
        By letting $A_i$ being the set of all natural numbers in the $i$-th row, we have satisfied the above conditions above.
    }
}