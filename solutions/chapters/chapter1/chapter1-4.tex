\section{Consequences of Completeness}

\bx{ % Exercise 1.4.1
    Recall that $\mathbb{I}$ stands for the set of irrational numbers. \ea{
        \item Show that if $a,b\in\mathbb{Q}$, then $ab$ and $a+b$ are elements of $\mathbb{Q}$ as well.
        \item Show that if $a\in\mathbb{Q}$ and $t\in\mathbb{I}$, then $a+t\in\mathbb{I}$ and $at\in\mathbb{I}$ as long as $a \neq 0$.
        \item Part (a) can be summarized by saying that $\mathbb{Q}$ is closed under addition and multiplication. Is $\mathbb{I}$ closed under addition and multiplication? Given two irrational numbers $s$ and $t$, what can we say about $s+t$ and $st$?
    }
    \sol{
        \ea{
            \item This is trivial. Since $a,b \in \mathbb{Q}$, they can be expressed as a fraction of two integers. Let $a = m/n$ and $b = x/y$ where $m,n,x,y\in\mathbb{Z}$, then $a+b = \frac{my+xn}{ny}$ and $ab = \frac{mx}{ny}$, which are fractions with integer numerators and denominators, hence $a+b$ and $ab$ are elements of $\mathbb{Q}$.
            \item \AFSOC that $a+t \in \mathbb{Q}$ and $at \in \mathbb{Q}$. Let $a+t=\alpha$ and $at=\beta$, so $t=\alpha-a$ and $t=\beta/a$. Since $\alpha, \beta, -a, 1/a \in \mathbb{Q}$, using part (a) gives use $t \in \mathbb{Q}$ which is a contradiction. Hence $a+t \in \mathbb{I}$ and $at\in \mathbb{I}$.
            \item $\mathbb{I}$ is not closed under addition or multiplication. Consider $\sqrt2 + (-\sqrt2) = 0$ and $\sqrt2 \cdot \sqrt2 = 2$.
        }
    }
}

\bx{ % Exercise 1.4.2
    Let $A \subseteq \mathbb{R}$ be nonempty and bounded above, and let $s \in \mathbb{R}$ have the property that for all $n\in \mathbb{N}$, $s+\frac{1}{n}$ is an upper bound for $A$ and $s-\frac1n$ is not an upperbound for $A$. SHow $s=\sup A$.

    \sol{
        We can rephrase Lemma 1.3.8 using the archimedean property. \er{
            \item \AFSOC $s<\sup A$, then there exists an $n$ such that $s+1/n < \sup A$, contradicting $\sup A$ being the least upper bound.
            \item \AFSOC $s>\sup A$, then there exists an $n$ such that $s-1/n > \sup A$ where $s-1/n$ is not an upper bound, contradicting $\sup A$ being an upper bound.
        }
        Hence, we can conclude that $\sup A = s$.
    }
}

\bx{ % Exercise 1.4.3
    Prove that $\bigcap^\infty_{n=1}(0, 1/n)=\emptyset$. Notice that this demonstrates that the intervals in the Nested Interval Property must be closed for the conclusion of the theorem to hold.
    \sol{
        \AFSOC the $x \in \bigcap^\infty_{n=1}(0, 1/n)$, so $0< x < 1/n$ for all $n \in \mathbb{N}$ which is impossible by archimedean property.
    }
}

\bx{ % Exercise 1.4.4
    Let $a < b$ be real numbers and consider the set $T = \mathbb{Q} \cap [a,b]$. Show $\sup T = b$.
    \sol{
        To show that $\sup T =b$, it needs to satify both conditions of supremum: \er{
            \item Since $x \leq b$ for all $x \in [a,b]$, $y \leq b$ for all $y \in T$ as $T \subseteq [a,b]$.
            \item \AFSOC $b'$ is also an upper bound such that $b'<b$. Since $\mathbb{Q}$ is dense in $\mathbb{R}$, there exists an $\alpha \in \mathbb{Q} \cap [b', b] \subseteq T$. This implies there exists $t \in T$ satisfying $b' < t$, which is a contradiction.
        }
    }
}

\bx{ % Exercise 1.4.5
    Using Exercise 1.4.1, supply a proof for Corollary 1.4.4 by considering real numbers $a - \sqrt{2}$ and $b - \sqrt{2}$.
    \sol{
        Recall that \textbf{Corollary 1.4.4} states that \textit{Given any two real numbers $a < b$, there exists an irrational number $t$ satisfying $a<t<b$.}
        \medskip

        \noindent Since $\mathbb{Q}$ is dense in $\mathbb{R}$, we can find $t \in \mathbb{Q}$ that is between any two real numbers $a-\sqrt{2}$ and $b-\sqrt2$, with $a<b$. Hence, $a-\sqrt2<t<b-\sqrt2$, meaning $a<t+\sqrt2<b$. By Exercise 1.4.1, $t+\sqrt2 \in \mathbb{I}$ and we are done.
    }
}

\bx{ %Exercise 1.4.6
    Recall that a set $B$ is \textit{dense} in $\mathbb{R}$ if can element $B$ can be found between any two real numbers $a<b$. WHich of the following sets are dense in $\mathbb{R}$? Take $p \in \mathbb{Z}$ and $q\in\mathbb{N}$ in every case. \ea{
        \item The set of all rational numbers $p/q$ with $q \leq 10$.
        \item The set of all rational numbers $p/q$ with $q$ a power of 2.
        \item The set of all rational numbers $p/q$ with $10|p| \geq q$.
    } 
    \sol{
        \ea{
            \item Not dense since we cannot make $|p|/q < 1/10$.
            \item Dense.
            \item Not dense since we cannot make $|p|/q < 1/10$.
        }
    }
}

\bx{ % Exercise 1.4.7
    Finish the proof of Theorem 1.4.5 by showing that the assumption $\alpha^2 > 2$ leads to a contradiction of the fact that $\alpha = \sup T$.
    \sol{
        Recall $T = \{t\in \mathbb{R}: t^2 < 2\}$ and $\alpha = \sup T$. \AFSOC $\alpha^2 > 2$, we will show that there exists an $n \in \mathbb{N}$ such that $(\alpha-1/n)^2 > 2$, contradicting the assumption that $\alpha$ is the least upper bound.

        Using $(\alpha - 1/n)^2 > 2$, we can find $n\in \mathbb{N}$ such that $(\alpha^2-1/n)>2$. \[
            2 < (\alpha - 1/n)^2 = \alpha^2 - \frac{2\alpha}{n} + 1/n^2 < \alpha^2 - \frac{2\alpha-1}{n}
        \]Then \[
            2 < \alpha^2 - \frac{2\alpha-1}{n} \implies n(2-\alpha^2) < 1 - 2\alpha
        \] Since $2-\alpha^2 < 0$, dividing reverses the inequality, giving \[
            n > \frac{1-2\alpha}{2-\alpha^2}
        \] 

        Hence we can pick $n \in \mathbb{N}$ such that $(\alpha^2-1/n) > 2$, so $\alpha$ is the the least upper bound which is a contradiction. Hence it is not possible for $\alpha^2 > 2$.
    }
}