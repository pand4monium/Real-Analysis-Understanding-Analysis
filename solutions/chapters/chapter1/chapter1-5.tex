\section{Cardinality}

\bx{ % Exercise 1.5.1
    Finish the following proof for Theorem 1.5.7.

    Assume $B$ is a countable set. Thus, there exists $f: \mathbb{N} \rightarrow B$, which is 1-1 and onto. Let $A \subseteq B$ be an infinite subset of $B$. We must show that $A$ is countable.

    Let $n_1=min\{n\in \mathbb{N}: f(n) \in A \}$. As a start to a definition of $g: \mathbb{N} \rightarrow A$, set $g(1) = f(n_1)$. Show how to inductively continue this process to produce a 1-1 function $g$ from $\mathbb{N}$ onto $A$.

    \sol{
        Let $n_k = min \{n\in \mathbb{N}: f(n) \in A, n \notin \{ n_1, n_2, \dots n_{k-1}\}\}$ and $g(k) = f(n_k)$. Since $g: \mathbb{N} \rightarrow A$ is 1-1 and onto, $A$ is countable.
    }
}

\bx{ % Exercise 1.5.2
    Review the proof of Theorem 1.5.6, part (ii) showing that $\mathbb{R}$ is uncountable, and then find the flow in the following erroneous proof that $\mathbb{Q}$ is uncountable:

    Assume, for contradiction, that $\mathbb{Q}$ is countable. Thus we can write $\mathbb{Q} = \{ r_1, r_2, r_3, \dots \}$ and, as before, construct a nested sequence of closed intervals with $r_n \notin I_n$. Our construction implies $\bigcap^\infty_{n=1} I_n = \emptyset$ while NIP implies $\bigcap^\infty_{n=1} \neq \emptyset$. This contradiction implies $\mathbb{Q}$ must therefore be uncountable.

    \sol{
        The Nested Interval Property only applies to $\mathbb{R}$ and not $\mathbb{Q}$.
    }
}

\bx{ % Exercise 1.5.3
    Use the following proofs for the statements in Theorem 1.5.8.\ea{
        \item First, prove statement (i) for two countable sets, $A_1$ and $A_2$. Example 1.5.3 (ii) may be a useful reference. Some technicalities can be avoided by first replacing $A_2$ with the set $B_2 = A_2 \backslash A_1 = \{x \in A_2: x \notin A_1 \}$. The point of this is that the union $A_1 \cup B_2$ is equal to $A_1 \cup A_2$ and the sets $A_1$ and $B_2$ are disjoint. (What happens if $B_2$ is finite?)
        
        Now, explain how that more general statement in (i) follows.
        \item Explain why induction cannot be used to prove part (ii) of Theorem 1.5.8 from part(i).
        \item Show how arranging $\mathbb{N}$ into the two-dimensional array \[ \begin{matrix}
            1 & 3 & 6 & 10 & 15 & \dots \\
            2 & 5 & 9 & 14 & \dots \\
            4 & 8 & 13 & \dots \\
            7 & 12 & \dots  \\
            \vdots \\
        \end{matrix} \] leads to a proof of Theorem 1.5.8 (ii).
    }

    \sol{\ea{
        \item Let $B = \{b_1, b_2, b_3, \dots \}$ and $C = \{c_1, c_2, c_3, \dot\}$ be countable, disjoint sets. We can define a function $g: \mathbb{N} \rightarrow B \cup C$ following a similar method of mapping $\mathbb{N}$ onto $\mathbb{Z}$, listing them as follows: \[
        B \cup C = \{b_1, c_1, b_2, c_2, \dots \}
        \] Implying that $B \cup C$ is countable. By letting $B = A_1$ and $C = A_2 \backslash A_1$, we can show that $A_1 \cup A_2$ is also countable.

        By using induction, suppose $A_1 \cup \dots \cup A_n$ is countable, $(A_1 \cup \dot \cup A_n) \cup A_{n+1}$ is the unions of two countable sets, which as proven above is also countable.

        \item Induction only shows something for each $n \in \mathbb{N}$, it does not apply in the infinite case. For example, $\bigcap^n_{k=1} [k, \infty) \neq \emptyset$ is true for all $n \in \mathbb{N}$, but the infinite case $\bigcap^\infty_{k=1} [k, \infty) \neq \emptyset$ is false.
        \item By rearranging $\mathbb{N}$ as in (c) gives us disjoint sets $C_n = \{k\in \mathbb{N}: k \text{ is in the n-th row}\}$, such that $\bigcup^\infty_{n=1}C_n = \mathbb{N}$. Let $B_n$ be disjoint sets, constructed as $B_1 = A_1$, $B_2 = A_2 \backslash A_1$, $\dots$, $B_n = A_n \backslash (A_1 \cup \dots A_{n-1})$. If we can find a function $f: C \rightarrow B$ that is bijective, we can show: \[
        f(\mathbb{N}) = f(\bigcup^\infty_{n=1}C_n) = \bigcup^\infty_{n=1} f_n(C_n) = \bigcup^\infty_{n=1} B_n = \bigcup^\infty_{n=1} A_n
        \] 

        Let $f_n: C_n \rightarrow B_n$ be bijective since $B_n$ is countable. Define $f: \mathbb{N} \rightarrow \bigcup_{n=1}^\infty B_n$ as \[
            f(n) = \begin{cases}
                f_1(n) & \text{ if } n \in C_1 \\
                f_2(n) & \text{ if } n \in C_2 \\
                \vdots \\
            \end{cases}
        \] We must now show that $f$ is bijective: \er{
            \item Since each $C_n$ is disjoint and each $f_n$ is 1-1, $f(n_1) = f(n_2) \implies n_1 = n_2$, meaning $f$ is 1-1.
            \item SInce any $b \in \bigcup^\infty_{n=1} B_n$ as $b\in B_n$ for some $n$, $b = f_n(x)$ has a solution since $f_n$ is onto. Letting $x = f^{-1}_n(b)$, we have $f(x) = f_n(x) =b$ as $f^{-1}_n(b) \in C_n$, meaning $f$ is onto.
        }
        By (i) and (ii), $f$ is bijective, so $\bigcup_{n=1}^\infty B_n$ is countable, implying that $\bigcup_{n=1}^\infty A_n$ is also countable, completing the proof.
    }}
}

\bx{ % Exercise 1.5.4
    \ea{
        \item Show $(a,b) \sim \mathbb{R}$ for any interval $(a, b)$.
        \item Show that an unbounded interval like $(a, \infty) = \{x : x> a\}$ has the same cardinality as $\mathbb{R}$ as well.
        \item Using open intervals makes it more convenient to produce the required 1-1, onto functions, but it is not really necessary. Show that $[0, 1)  \sim  (0, 1)$ by exhibiting a 1-1 onto function between the two sets.
    }

    \sol{
        \ea{
            \item Example 1.5.4 provides the function $f(x) = \frac{x}{x^2-1}$ which takes the interval $(-1, 1)$ onto $\mathbb{R}$ in a 1-1 fashion. If we shift $f$ to the interval of $(a, b)$, we get: \[
            g(x) = f(\frac{2x-1}{b-a} - a)
            \] which maps $(a, b)$ onto $\mathbb{R}$ in a 1-1 fashion.

            \item To show that $(a, \infty)  \sim  \mathbb{R}$, we need to find a function $h$ that maps $\mathbb{R}$ onto $(a, \infty)$. Consider the function $h(x) = e^x + a$, which maps $\mathbb{R}$ onto $(a, \infty)$ in a 1-1 fashion. Hence we are done.
            \item Define $f: [0,1) \rightarrow (0, 1)$ as \[
                f(x) = \begin{cases}
                    1/2 & \text{ if } x = 0 \\
                    1/4 & \text{ if } x = 1/2 \\
                    1/8 & \text{ if } x = 1/4 \\
                    \vdots & \\
                    x & \text{ otherwise }
                \end{cases}
            \] To show that both $[0,1)$ and $(0,1)$ have the both cardinality, we need to prove that $f$ is 1-1 and onto.

            \noindent We start by showing that $y = f(x)$ has exactly one solution for all $y \in (0,1)$.

            \noindent If $y = 1/2^n$, then the only solution is $y = f(1/2^{n-1})$ (or $x = 0$ in the special case $n=1$).
            
            \noindent Otherwise, the only solution is $y=f(y)$.
        }
    }
}

\bx{ %Exercise 1.5.5
    \ea{
        \item Why is $A \sim A$ for every set $A$?
        \item Given sets $A$ and $B$, explain why $A \sim B$ is equivalent to asserting $B \sim A$.
        \item For three sets $A$, $B$, and $C$, show that $A \sim B$ and $B \sim C$ implies $A \sim C$. These three properties are hwat is meant by saying that $ \sim $ is an \textit{equivalence relation}.
    }
    \sol{\ea{
        \item The identity function $f(x) = x$ is trivially bijective.
        \item If $f: A \rightarrow B$ is bijective, then the inverse $f^{-1}: B \rightarrow A$ is also bijective.
        \item If $f: A \rightarrow B$ and $g: B \rightarrow C$, then $g \circ f: A \rightarrow C$ is a bijection, thus $A \sim C$.
    }}
}

\bx{ % Exercise 1.5.6
    \ea{
        \item Give an example of a countable collection of disjoint open intervals.
        \item Give an example of an uncountable collection of disjoint open intervals, or argue that no such collection exists.
    }
    \sol{
        \ea{
            \item Consider $I_1 = (1, 2)$, $I_2 = (2, 3)$, $\dots$, $I_n = (n, n+1)$.
            \item No such collection exists. Let $A$ denote this set. 
            
            \noindent For any nonempty interval $I_n$, since $\mathbb{Q}$ is dense in $\mathbb{R}$, we can find an $r\in \mathbb{Q}$ such that $r \in I_n$. Assigning each $I \in A$ a rational number $r \in I$ proves that $I \subseteq \mathbb{Q}$, thus $I$ is countable.
        }
    }
}

\bx{ % Exercise 1.5.7
    Consider the open interval $(0,1)$, and let $S$ be the set of points in the open unit square; that is, $S = \{(x,y): 0< x,y< 1 \}$. \ea{
        \item Find a 1-1 function that maps $(0,1)$ into, but not necessarily onto, $S$. (This is easy)
        \item Use the fact that every real number has a decimal expansion to produce a 1-1 function that maps $S$ into $(0, 1)$. Discuss whether the formulated function is onto. (Keep in mind that any terminating decimal expansion such as $.235$ represents the same real number as $.234999 \dots$.)
    } The Schröder-Bernstein Theorem discussed in Exercise 1.5.11 can now be applied to conclude that $(0,1)  \sim  S$.
    \sol{\ea{
        \item Consider $f(x) = (\frac14x, \frac13)$.
        \item Let $g: S \rightarrow (0,1)$ be a function that interweaves decimals in the representation without trailing nines, padding with zeros if necessary. $g(0.12, 0.34) = 0.1324$, $g(0.1\bar{9}, 0.2) = g(0.2, 0.2) = 0.22$, $g(0.4, 0.89) = 0.4809$, $g(0.6, 0.\bar{7}) = 0.67\overline{07}$, etc.
        \medskip

        \noindent To prove that $g$ is 1-1, we need show that there does not exists any solutions for $g(x_1, y_1) = g(x_2, y_2)$. Every real number can be written with tow representations, one with trailing 9's and one without. However, $g(x, y) = 0.d_1d_2d_3\dots \bar{9}$ is impossible as it would imply both $x$ and $y$ has trailing 9's, which contradicts the definition of $g$. Therefore, $g(s)$ is unique and so $g$ is 1-1.
        \medskip

        \noindent $g$ is not onto since $g(x,y) = 0.1$ has no solutions.
    }}
}

\bx{ % Exercise 1.5.8
    Let $B$ be a set of positive real numbers with the property that adding together any finite subset of elements from $B$ will always give a sum of 2 or less. Show $B$ must be finite or countable.
    \sol{
        It is obvious that $B \cap [a, 2)$ must be finite for some $a \in (0, 2)$, otherwise, we can choose $\lceil \frac2a \rceil$ number of elements from $B \cap [a, 2)$ would result in a sum of more than 2. Since $B$ is the countable union of finite sets $\bigcup^\infty_{n=1} B \cap [\frac1n , 2)$, $B$ must be countable or finite.
    }
} 