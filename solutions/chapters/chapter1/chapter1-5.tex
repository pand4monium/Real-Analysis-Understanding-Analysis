\section{Cardinality}

\bx{ % Exercise 1.5.1
    Finish the following proof for Theorem 1.5.7.

    Assume $B$ is a countable set. Thus, there exists $f: \mathbb{N} \rightarrow B$, which is 1-1 and onto. Let $A \subseteq B$ be an infinite subset of $B$. We must show that $A$ is countable.

    Let $n_1=min\{n\in \mathbb{N}: f(n) \in A \}$. As a start to a definition of $g: \mathbb{N} \rightarrow A$, set $g(1) = f(n_1)$. Show how to inductively continue this process to produce a 1-1 function $g$ from $\mathbb{N}$ onto $A$.

    \sol{
        Let $n_k = min \{n\in \mathbb{N}: f(n) \in A, n \notin \{ n_1, n_2, \dots n_{k-1}\}\}$ and $g(k) = f(n_k)$. Since $g: \mathbb{N} \rightarrow A$ is 1-1 and onto, $A$ is countable.
    }
}

\bx{ % Exercise 1.5.2
    Review the proof of Theorem 1.5.6, part (ii) showing that $\mathbb{R}$ is uncountable, and then find the flow in the following erroneous proof that $\mathbb{Q}$ is uncountable:

    Assume, for contradiction, that $\mathbb{Q}$ is countable. Thus we can write $\mathbb{Q} = \{ r_1, r_2, r_3, \dots \}$ and, as before, construct a nested sequence of closed intervals with $r_n \notin I_n$. Our construction implies $\bigcap^\infty_{n=1} I_n = \emptyset$ while NIP implies $\bigcap^\infty_{n=1} \neq \emptyset$. This contradiction implies $\mathbb{Q}$ must therefore be uncountable.

    \sol{
        The Nested Interval Property only applies to $\mathbb{R}$ and not $\mathbb{Q}$.
    }
}