\section{The Axiom of Completeness}

\bx{ % Exercise 1.3.1
    \ea{
        \item Write a formal definition in the style of Definition 1.3.2 for the \textit{infimum} or \textit{greatest lower bound} of a set.
        \item Now, state and prove a version of Lemma 1.3.8 for greatest lower bound.
    }
    \sol{
        \ea{
            \item A real number $i$ is the \textit{greatest upper bound} for a set $A \subseteq \mathbb{R}$ if it meets the following two criteria: \er{
                \item $i$ is a lower bound of A;
                \item if $b$ is any lower bound for A, then $i \geq b$.
            }
            \item Assume $i \in \mathbb{R}$ is a lower bound for a set $A \subseteq \mathbb{R}$. Then $i = \inf A$ \textit{if and only if}, for every choice of $\epsilon > 0$, there $\exists a \in A$ satisfying $i + \epsilon > a$.
            \begin{proof}
                Rephrasing the lemma gives us: Given that $i$ is a lower bound, $i$ is the greatest lower bound if and only if any number greater than $i$ is not a lower bound. \er{
                    \item Suppose $i = \inf A$ and consider $i+\epsilon$ for an arbituarily chosen $\epsilon > 0$. Since $i + \epsilon > i$, part (ii) of the definition impliers that $i + \epsilon$ is not a lower bound for $A$. If this is the case, then there must be some element $a \in A$ such that $i + \epsilon > a$.
                    \item Conversely, assume $i$ is a lower bound with the property that for any $\epsilon > 0$, $s+\epsilon$ is not a lower bound of A. Note that this implies that if $b$ is any number more than $i$, then $b$ is not an upperbound. Since we have argued that any larger number than $i$ cannot be a lower bound, if $b$ is some other upper bound for $A$, then $i \geq b$.
                }
            \end{proof}
        }
    }
}

\bx{ % Exercise 1.3.2
    Give an example of each of the following, or state that the request is impossible. \ea{
        \item A set $B$ with $\inf B \geq \sup B$.
        \item A finite set that contains its infimum but not its supremum.
        \item A bounded subset of $\mathbb{Q}$ that contains its supremum but not its infimum.
    }
    \sol{
        \ea{
            \item Possible. Consider the set ${0}$, where $\inf \{0\} = \sup \{0\} = 0$.
            \item Not possible as all finite sets must contain its supremum and infimum.
            \item Possible. Consider $A = \{r \in \mathbb{Q} \hspace{1ex} | \hspace{1ex}1<r\leq2\}$.
        }
    }
}

\bx{ % Exercise 1.3.3
    \ea{
        \item Let $A$ be nonempty and bounded below, and define $B=\{b\in \mathbb{R}: b \text{ is a lower bound for } A\}$. Show that $\sup B = \inf A$.
        \item Use (a) to explain why there is no need to assert that greatest lower bounds exist as part of the Axiom of Completeness.
    }
    \sol{
        \ea{
            \item By definition, $\sup B$ is the greatest lower bound for $A$, meaning it equals $\inf A$.
            \item Part (a) proves the greatest lower bound exists using the least upper bound.
        }
    }
}

\bx{ %Exercise 1.3.4
    Let $A_1, A_2, A_3, \dots$ be a collection of nonempty sets, each of which is bounded above.
    \ea{
        \item Find a formula for $\sup(A_1\cup A_2)$. Extend this to $\sup (\bigcup^n_{k=1}A_k)$.
        \item Consider $\sup (\bigcup^\infty_{k=1} A_k)$. Does the formula in (a) extend to the infinite case?
    }
    \sol{
        \ea{
            \item $\sup (A_1 \cup A_2) = \sup\{\sup A_1, \sup A_2\}$ and $\sup(\bigcup^n_{k=1}A_k) = \sup\{\sup A_k \hspace{1ex} | \hspace{1ex} k = 1, \dots, n \}$
            \item This formula does not extend to infinity. Consider $A_k = [k, k+1]$, where $\bigcup^\infty_{k=1}A_k$ is unbounded.
        }
    }
}

\bx{ % Exercise 1.3.5
    As in Example 1.3.7, let $A\subseteq\mathbb{R}$ be nonempty and bounded above, and let $c \in \mathbb{R}$. This time define the set $cA = \{ca: a \in A\}$.
    \ea{
        \item If $c \geq 0$, show that $\sup(cA) = c \sup A$.
        \item Prostulate a similar type of statement for $\sup(cA)$ for the case $c<0$.
    }

    \sol{
        \ea{
            \item The case of $c=0$ is trivial as it implies that $cA = \{0\}$. Hence $\sup(cA)=c\sup A = 0$.
            
            For $c > 0$, we need to show that $c \sup A$ is the lowest upper bound. Assume $c > 0$. Let $s = c \sup A$. Suppose $ca > s$, then $a > \sup A$ which is impossible, meaning that $s$ is an upper bound on $cA$. Now suppose $s'$ is an upper bound on $cA$ and $s' < s$. Then $s'/c < s/c = \sup A$, meaning $s'/c$ cannot bound $A$. Hence there $\exists a \in A$ such that $s'/c < a$, meaning $s' > ca$, thus $s'$ cannot be an upper bound on $cA$, so $s = c \sup A$ is the least upper bound.
            \item $\sup (cA) = c \inf A$ for $ c < 0$
        }
    }
}

\bx{ % Exercise 1.3.6
    Given sets $A$ and $B$, define $A + B = \{ a+ b: a\in A \text{ and } b \in B\}$. Follow these steps to prove that if $A$ and $B$ are nonempty and bounded above then $\sup (A+B) = \sup A + \sup B$.
    \ea{
        \item Let $s=\sup A$ and $t=\sup B$. Show $s+t$ is an upper bound for $A+B$.
        \item Now let $u$ be an arbituary upper bound for $A+B$, and temporarily fix $a \in A$. Show $t \leq u+a$.
        \item Finally, show $\sup (A+B) = s+t$.
        \item Construct another proof of this same fact using Lemma 1.3.8.
    }

    \sol{
        \ea{
            \item By definition of supremum, $a \leq s$ and $b \leq t$. Adding both equations give $a + b \leq s + t$, hence $s+t$ is an upper bound.
            \item Since $a+b \leq u$ implies $b \leq u-a$, $u-a$ is an upper bound on b, meaning it is greater or equal to the least upper bound of $t$, giving $t \leq u-a$.
            \item From (a), we have shown that $s+t$ is an upper bound for $A+B$, hence it is sufficent to show that $s+t$ is the least upper bound.
            
            Let $u = \sup (A+B)$, from (b) we have $t \leq u - a$ and $s \leq u - b$. Adding and rearranging gives $a + b \leq 2u - s - t$. Since $2u-s-t$ is an upper bound on $A+B$, it must be greater or equal to the least upper bound, giving $u \leq 2u-s-t$, implying $s+t \leq u$. Since $u$ is the least upper bound, $s+t$ must equal $u$.
            \item Showing $s+t-\epsilon$ is not an upper bound for any $\epsilon>0$ proves that it is the least upper bound by Lemma 1.3.8. Rearranging gives $(s-\epsilon/2)+(t-\epsilon/2)$ we know $\exists a > (s-\epsilon/2)$ and $b > (t-\epsilon/2)$, therefore $a+b > s+t-\epsilon$, meaning $s+t$ cannot be made smaller and thus is the least upper bound.
        }
    }
}

\bx{ % Exercise 1.3.7
    Prove that if $a$ is an upper bound for $A$, and if $a$ is also an element of $A$, then it must be that $a = \sup A$.
    \sol{
        Since it is given that $a$ is an upper bound for $A$, we just have to show that $a$ is the least upper bound, meaning any number lower than $a$ would have an $a' \in A$ such that $a' > a$.
        \medskip

        \noindent Suppose $a-\epsilon$ is also an upper bound for $A$ for some $\epsilon > 0$. This is not possible has $a > a'$ and $a \in  A$. Hence by contradiction, $a$ is the lowest upper bound, meaning $a = \sup A$.
    }
}