\section{The Axiom of Completeness}

\bx{
    \ea{
        \item Write a formal definition in the style of Definition 1.3.2 for the \textit{infimum} or \textit{greatest lower bound} of a set.
        \item Now, state and prove a version of Lemma 1.3.8 for greatest lower bound.
    }
    \sol{
        \ea{
            \item A real number $i$ is the \textit{greatest upper bound} for a set $A \subseteq \mathbb{R}$ if it meets the following two criteria: \er{
                \item $i$ is a lower bound of A;
                \item if $b$ is any lower bound for A, then $i \geq b$.
            }
            \item Assume $i \in \mathbb{R}$ is a lower bound for a set $A \subseteq \mathbb{R}$. Then $i = \inf A$ \textit{if and only if}, for every choice of $\epsilon > 0$, there $\exists a \in A$ satisfying $i + \epsilon > a$.
            \begin{proof}
                Rephrasing the lemma gives us: Given that $i$ is a lower bound, $i$ is the greatest lower bound if and only if any number greater than $i$ is not a lower bound. \er{
                    \item Suppose $i = \inf A$ and consider $i+\epsilon$ for an arbituarily chosen $\epsilon > 0$. Since $i + \epsilon > i$, part (ii) of the definition impliers that $i + \epsilon$ is not a lower bound for $A$. If this is the case, then there must be some element $a \in A$ such that $i + \epsilon > a$.
                    \item Conversely, assume $i$ is a lower bound with the property that for any $\epsilon > 0$, $s+\epsilon$ is not a lower bound of A. Note that this implies that if $b$ is any number more than $i$, then $b$ is not an upperbound. Since we have argued that any larger number than $i$ cannot be a lower bound, if $b$ is some other upper bound for $A$, then $i \geq b$.
                }
            \end{proof}
        }
    }
}