\section{Some Preliminaries}

\bx{ %Exercise 1.2.1
    \ea{
        \item Prove that $\sqrt{3}$ is irrational. Does a similar argument work to show $\sqrt{6}$ is rational?
        \item Where does the proof break down if we try to prove $\sqrt{4}$ is irrational?
    }
    \sol{
    \ea{
        \item 

        \begin{proof}
            \AFSOC that $\sqrt{3}$ is rational, so $\exists m, n \in \mathbb{Z}$, such that 
            \begin{equation*}
                \sqrt{3} = \frac{m}{n},
            \end{equation*}
            where $\frac{m}{n}$ is in the lowest reduced terms. By squaring both sides, we obtain $3 = (\frac{m}{n})^2 \implies 3n^2 = m^2$. Now, we know that $m^2$ is a multiple of 3 and thus $m$ must also be a multiple of 3. We can then write $m=3k$, deriving 
            \begin{align*}
                (\sqrt3)^2 &= \pa{\frac{3k}{n}}^2 \\
                3n^2&=9k^2 \\
                n^2 &= 3k^2
            \end{align*}

            Similar to above, we can conclude that n is a multiple of 3. However this is a contradiction since $m,n$ are both multiples of 3 but we assumed that $\frac{m}{n}$ was in its lowest reduced term. Thus we conclude that $\sqrt3$ is irrational.
        \end{proof}
        The same proof for $\sqrt3$ works for $\sqrt6$ as well.

            \item We cannot conclude that $\sqrt{4} = \frac{m}{n}$ imply that $m$ is a multiple of $4$, as we have 
        \begin{equation*}
            4n^2 = m^2 \implies 2n=m,
        \end{equation*}
        preventing us from reaching our contradiction that $m/n$ is not in its lowest terms.
    }
    }
}


\bx{ %Exercise 1.2.2
    Show that there is no rational number $r$ satisfying $2^r = 3$.
    \sol{
        \begin{proof}
            If $r=0$, then $2^r=1\neq3$. Suppose $r=p/q$ to get $2^p = 3^q$, which is not possible as $2$ and $3$ share no common factors. Hence $r$ is not rational.
        \end{proof}
    }
}


\bx{ %Exercise 1.2.3
    Decide which of the following represent true statements about the nature of sets. For any that are false, provide a specific example where the statement in question does not hold.
    \ea{
        \item If $A_1\supseteq A_2\subseteq A_3\subseteq A_4 \dots$ are all sets containing an infinite number of elements, then the intersections $\bigcap^\infty_{n=1}A_n$ is infinite as well.
        \item If $A_1\supseteq A_2\subseteq A_3\subseteq A_4 \dots$ are all finite, nonempty sets of real numbers, then the intersection $\bigcap^\infty_{n=1}A_n$ is finite and non-empty.
        \item $A\cap(B\cup C) = (A \cap B) \cup C$.
        \item $A \cap (B \cap C) = (A \cap B) \cap C$.
        \item $A \cap (B \cup C) = (A \cap B) \cup (A \cap C)$.
    }
    \sol{
        \ea{
            \item False. Consider $A_n = \{n, n+1, n+2, \dots\}$, then $\bigcap^\infty_{n=1}A_n = \emptyset$.
            \item True. Since all $A_n$ are nonempty, $\exists n \in \mathbb{N}$ such that $A_n = \{x\}$ for some real $x$. Hence $\bigcap^\infty_{n=1}A_n \subseteq \{x\}$ which is empty. Since $A_1$ is finite, $\bigcap^\infty_{n=1}A_n \subseteq \{x\} \subset A_1$ is finite.
            \item False. If $A=\emptyset$, then $\emptyset = C$
            \item True. Intersection is associative as evident that both LHS and RHS implies the $x \in A, B, C$
            \item True. Drawing a Venn Diagram illustrates this.
        }
    }
}


\bx{ %Exercise 1.2.4
    Produce an infinite collection of sets $A_1, A_2, A_3, \dots$ with the property that every $A_i$ has an infinite number of elements, $A_i \cap A_j = \emptyset$ for all $i \neq j$, and $\bigcup^\infty_{i=1}A_i = \mathbb{N}$.

    \sol{
        Consider arranging the elements of $\mathbb{N}$ in a square as such.
        \[
        \begin{matrix}
            1 & 3 & 6 & 10 & 15 & \dots \\
            2 & 5 & 9 & 14 & \dots & \\
            4 & 8 & 13 & \dots & & \\
            7 & 12 & \dots & & & \\
            11 & \dots & & & & \\
            \vdots & & & & & \\
        \end{matrix}
        \]
        By letting $A_i$ being the set of all natural numbers in the $i$-th row, we have satisfied the above conditions above.
    }
}


\bx{ %Exercise 1.2.5
    \textbf{(De Morgan's Law)} Let $A$ and $B$ be subsets of $\mathbb{R}$.
    \ea{
        \item If $x \in (A \cap B)^c$, explain why $x \in A^c \cup B^c$. THis shows that $(A \cap B)^c \subseteq A^c \cup B^c$.
        \item Prove the reverse inclusion $(A \cap B)^c \sqsubseteq A^c \cup B^c$, and conclude that $(A \cap B)^c = A^c \cup B^c$.
        \item Show $(A \cup B)^c = A^c \cup B^c$ by demonstrating inclusion both ways.
    }

    \sol{
        \ea{
            \item If $x \in (A \cap B)^c$, then $x \notin A \cap B$, so $x \notin A$ or $x \notin B$, implying $x \in A^c$ or $x \in B^c$, therfore $x \in A^c \cup B^c$.
            \item If $x \in A^c \cup B^c$, then $x \in A^c$ or $x \in B^c$, so $x \notin A$ and $x \notin B$, implying $x \notin A \cap B$, therefore $x \in (A \cap B)^c$.
            Since $(A \cap B)^c \subseteq A^c \cup B^c$ and $(A \cap B)^c \sqsubseteq A^c \cup B^c$, we can conclude that both sets are equal.
            \item To show that $(A \cap B)^c = A^c \cup B^c$, we need to demonstrate inclusion both ways. \er{
                \item If $x \in (A \cup B)^c$, then $x \notin A \cup B$, so $x \notin A$ or $x \notin B$, implying $x \in A^c$ or $x \in B^c$, therefore $x \in A^c \cup B^c$.
                \item If $x \in A^c \cap B^c$, then $x \in A^c$ and $x \in B^c$, so $x \notin A$ and $x \notin B$, implying $x \notin A \cup B$, which is just $x \in (A \cup B)^c$.
            }
        }
    }
}

\bx{ %Exercise 1.2.6
    \ea{
        \item Verify the triangle inequality in the special case where $a$ and $b$ have the same sign.
        \item Find an efficient proof for all the cases at once by first demonstrating $(a+b)^2 \leq (|a|+|b|)^2$.
        \item Prove $|a-b| \leq |a-c|+|c-d|+|d-b|$ for all $a$, $b$, $c$ and $d$.
        \item Prove $||a|-|b|| \leq |a-b|$. (The unremarkable identity $a = a - b + b$ may be useful.)
    }
    \sol{
        \ea{
        \item With both $a$ and $b$ having the same sign, then $|a| + |b| = |a+b|$, which satisfies $|a| + |b| \geq |a+b|$.
        \item Note that $(a+b)^2 \leq (|a|+|b|)^2$ reduces to $ab \leq |a||b|$, which is true as LHS can be negative while RHS cannot. Since squaring preserves inequality, this implies that $|a+b| \leq |a| + |b|$.
        \item Notice that $a - b = (a-c) + (c-d) + (d-b)$. Hence by triangle inequality, \[
        |a-b| = |(a-c)+(c-d)+(d-b)|\leq |a-c| + |c-d| + |d-b|
        \] for all $a$, $b$, $c$ and $d$.
        \item Since $||a|-|b|| = ||b|-|a||$, WLOG, we can assume that $|a| \ge |b|$. Then \[
            ||a|-|b|| = |a| - |b| = |(a-b)+b|-|b| \leq |a-b| + |b| - |b| = |a-b|
        \]
        }
    }
}

bx{%Exercise 1.2.7
    Given a function $f$ and a subset $A$ of its domain, let $f(A)$ represent the range of $f$ over the set $A$; that is, $f(A) = \{f(x): x\in A\}$.
    \ea{
        \item Let $f(x)=x^2$. If $A = [0,2]$ (the closed interval $\{x \in \mathbb{R}: 0 \leq x \leq 2 \}$) and $B = [1,4]$, find $f(A)$ and $f(B)$. Does $f(A\cap B) = f(A)\cap f(B)$ in this case? Does $f(A\cup B)=f(A) \cup f(B)$?
        \item Find two sets $A$ and $B$ for which $f(A\cap B) \neq f(A) \cap f(B)$.
        \item Show that, for an arbituary function $g: \mathbb{R} \rightarrow \mathbb{R}$, it is always true that $g(A \cap B) \subseteq g(A) \cap g(B)$ for all sets $A,B \subseteq \mathbb{R}$.
        \item Form and prove a conjecture about the relationship between $g(A \cup B)$ and $g(A) \cup g(B)$ for an arbituary function $g$.
    }

    \sol{
        \ea{
            \item For $f(x)=x^2$, $f(A)=f([0,2])=[0,4]$ and $f(B) = f([1,4]) = [1, 16]$. \begin{align*}
                f(A\cap B) = f([0,2]\cap [1, 4]) = f([1, 2]) = [1,&4] = [0, 4] \cap [1, 16] = f([1, 2]) \cap f([2, 4]) =f(A) \cap f(B) \\
                f(A \cup B) = f([0,2]\cup [1,4]) = f([0,4]) = [0,&16] = [0, 4] \cup [1, 16] = f([0,2]) \cup f([1, 4]) = f(A) \cup f(B)
            \end{align*}
            \item Consider $A = [0, 2]$ and $B = [-2, 0]$. $f(A \cap B) = \{0\}$, but $f(A) \cap f(B) = [0, 4]$.
            \item Suppose $y \in g(A \cap B)$, then $\exists x \in A \cap B$ such that $g(x) = y$. This implies that $x \in A$ and $x \in B$, so $x \in A \cap B$, hence $y \in g(A \cap B)$. Note that contrary may not always be true as it is possible for $x_1 \in A \backslash B$ and $x_2 \in B \backslash A$ such that $g(x_1) = g(x_2)$.
            \item I conjecture that $g(A \cup B) = g(A) \cup g(B)$. To prove this, we have to show inclusion both ways: \er{
                \item Let $y \in g(A\cup B)$, then $\exists x \in A \cup B$ such that $y=g(x)$. This implies that $x \in A$ or $x \in B$, so $y \in g(A)$ or $y \in g(B)$, hence $y \in g(A) \cup g(B)$.
                \item Let $y \in g(A) \cup g(B)$, then $y \in g(A)$ or $y \in g(B)$, implying $x \in A$ or $x \in B$ such that $y =g(x)$. So $x \in A \cup B$, hence $y \in g(A \cup B)$.
            }
        }
    }
}