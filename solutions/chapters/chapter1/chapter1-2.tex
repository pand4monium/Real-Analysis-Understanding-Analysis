\section{Some Preliminaries}

\bx{ %Exercise 1.2.1
    \ea{
        \item Prove that $\sqrt{3}$ is irrational. Does a similar argument work to show $\sqrt{6}$ is rational?
        \item Where does the proof break down if we try to prove $\sqrt{4}$ is irrational?
    }
    \sol{
    \ea{
        \item 

        \begin{proof}
            \AFSOC that $\sqrt{3}$ is rational, so $\exists m, n \in \mathbb{Z}$, such that 
            \begin{equation*}
                \sqrt{3} = \frac{m}{n},
            \end{equation*}
            where $\frac{m}{n}$ is in the lowest reduced terms. By squaring both sides, we obtain $3 = (\frac{m}{n})^2 \implies 3n^2 = m^2$. Now, we know that $m^2$ is a multiple of 3 and thus $m$ must also be a multiple of 3. We can then write $m=3k$, deriving 
            \begin{align*}
                (\sqrt3)^2 &= \pa{\frac{3k}{n}}^2 \\
                3n^2&=9k^2 \\
                n^2 &= 3k^2
            \end{align*}

            Similar to above, we can conclude that n is a multiple of 3. However this is a contradiction since $m,n$ are both multiples of 3 but we assumed that $\frac{m}{n}$ was in its lowest reduced term. Thus we conclude that $\sqrt3$ is irrational.
        \end{proof}
        The same proof for $\sqrt3$ works for $\sqrt6$ as well.

            \item We cannot conclude that $\sqrt{4} = \frac{m}{n}$ imply that $m$ is a multiple of $4$, as we have 
        \begin{equation*}
            4n^2 = m^2 \implies 2n=m,
        \end{equation*}
        preventing us from reaching our contradiction that $m/n$ is not in its lowest terms.
    }
    }
}


\bx{ %Exercise 1.2.2
    Show that there is no rational number $r$ satisfying $2^r = 3$.
    \sol{
        \begin{proof}
            If $r=0$, then $2^r=1\neq3$. Suppose $r=p/q$ to get $2^p = 3^q$, which is not possible as $2$ and $3$ share no common factors. Hence $r$ is not rational.
        \end{proof}
    }
}


\bx{ %Exercise 1.2.3
    Decide which of the following represent true statements about the nature of sets. For any that are false, provide a specific example where the statement in question does not hold.
    \ea{
        \item If $A_1\supseteq A_2\subseteq A_3\subseteq A_4 \dots$ are all sets containing an infinite number of elements, then the intersections $\bigcap^\infty_{n=1}A_n$ is infinite as well.
        \item If $A_1\supseteq A_2\subseteq A_3\subseteq A_4 \dots$ are all finite, nonempty sets of real numbers, then the intersection $\bigcap^\infty_{n=1}A_n$ is finite and non-empty.
        \item $A\cap(B\cup C) = (A \cap B) \cup C$.
        \item $A \cap (B \cap C) = (A \cap B) \cap C$.
        \item $A \cap (B \cup C) = (A \cap B) \cup (A \cap C)$.
    }
    \sol{
        \ea{
            \item False. Consider $A_n = \{n, n+1, n+2, \dots\}$, then $\bigcap^\infty_{n=1}A_n = \emptyset$.
            \item True. Since all $A_n$ are nonempty, $\exists n \in \mathbb{N}$ such that $A_n = \{x\}$ for some real $x$. Hence $\bigcap^\infty_{n=1}A_n \subseteq \{x\}$ which is empty. Since $A_1$ is finite, $\bigcap^\infty_{n=1}A_n \subseteq \{x\} \subset A_1$ is finite.
            \item False. If $A=\emptyset$, then $\emptyset = C$
            \item True. Intersection is associative as evident that both LHS and RHS implies the $x \in A, B, C$
            \item True. Drawing a Venn Diagram illustrates this.
        }
    }
}


\bx{ %Exercise 1.2.4
    Produce an infinite collection of sets $A_1, A_2, A_3, \dots$ with the property that every $A_i$ has an infinite number of elements, $A_i \cap A_j = \emptyset$ for all $i \neq j$, and $\bigcup^\infty_{i=1}A_i = \mathbb{N}$.

    \sol{
        Consider arranging the elements of $\mathbb{N}$ in a square as such.
        \[
        \begin{matrix}
            1 & 3 & 6 & 10 & 15 & \dots \\
            2 & 5 & 9 & 14 & \dots & \\
            4 & 8 & 13 & \dots & & \\
            7 & 12 & \dots & & & \\
            11 & \dots & & & & \\
            \vdots & & & & & \\
        \end{matrix}
        \]
        By letting $A_i$ being the set of all natural numbers in the $i$-th row, we have satisfied the above conditions above.
    }
}


\bx{ %Exercise 1.2.5
    \textbf{(De Morgan's Law)} Let $A$ and $B$ be subsets of $\mathbb{R}$.
    \ea{
        \item If $x \in (A \cap B)^c$, explain why $x \in A^c \cup B^c$. THis shows that $(A \cap B)^c \subseteq A^c \cup B^c$.
        \item Prove the reverse inclusion $(A \cap B)^c \sqsubseteq A^c \cup B^c$, and conclude that $(A \cap B)^c = A^c \cup B^c$.
        \item Show $(A \cup B)^c = A^c \cup B^c$ by demonstrating inclusion both ways.
    }

    \sol{
        \ea{
            \item If $x \in (A \cap B)^c$, then $x \notin A \cap B$, so $x \notin A$ or $x \notin B$, implying $x \in A^c$ or $x \in B^c$, therfore $x \in A^c \cup B^c$.
            \item If $x \in A^c \cup B^c$, then $x \in A^c$ or $x \in B^c$, so $x \notin A$ and $x \notin B$, implying $x \notin A \cap B$, therefore $x \in (A \cap B)^c$.
            Since $(A \cap B)^c \subseteq A^c \cup B^c$ and $(A \cap B)^c \sqsubseteq A^c \cup B^c$, we can conclude that both sets are equal.
            \item To show that $(A \cap B)^c = A^c \cup B^c$, we need to demonstrate inclusion both ways. \er{
                \item If $x \in (A \cup B)^c$, then $x \notin A \cup B$, so $x \notin A$ or $x \notin B$, implying $x \in A^c$ or $x \in B^c$, therefore $x \in A^c \cup B^c$.
                \item If $x \in A^c \cap B^c$, then $x \in A^c$ and $x \in B^c$, so $x \notin A$ and $x \notin B$, implying $x \notin A \cup B$, which is just $x \in (A \cup B)^c$.
            }
        }
    }
}

\bx{ %Exercise 1.2.6
    \ea{
        \item Verify the triangle inequality in the special case where $a$ and $b$ have the same sign.
        \item Find an efficient proof for all the cases at once by first demonstrating $(a+b)^2 \leq (|a|+|b|)^2$.
        \item Prove $|a-b| \leq |a-c|+|c-d|+|d-b|$ for all $a$, $b$, $c$ and $d$.
        \item Prove $||a|-|b|| \leq |a-b|$. (The unremarkable identity $a = a - b + b$ may be useful.)
    }
    \sol{
        \ea{
        \item With both $a$ and $b$ having the same sign, then $|a| + |b| = |a+b|$, which satisfies $|a| + |b| \geq |a+b|$.
        \item Note that $(a+b)^2 \leq (|a|+|b|)^2$ reduces to $ab \leq |a||b|$, which is true as LHS can be negative while RHS cannot. Since squaring preserves inequality, this implies that $|a+b| \leq |a| + |b|$.
        \item Notice that $a - b = (a-c) + (c-d) + (d-b)$. Hence by triangle inequality, \[
        |a-b| = |(a-c)+(c-d)+(d-b)|\leq |a-c| + |c-d| + |d-b|
        \] for all $a$, $b$, $c$ and $d$.
        \item Since $||a|-|b|| = ||b|-|a||$, WLOG, we can assume that $|a| \ge |b|$. Then \[
            ||a|-|b|| = |a| - |b| = |(a-b)+b|-|b| \leq |a-b| + |b| - |b| = |a-b|
        \]
        }
    }
}

\bx{%Exercise 1.2.7
    Given a function $f$ and a subset $A$ of its domain, let $f(A)$ represent the range of $f$ over the set $A$; that is, $f(A) = \{f(x): x\in A\}$.
    \ea{
        \item Let $f(x)=x^2$. If $A = [0,2]$ (the closed interval $\{x \in \mathbb{R}: 0 \leq x \leq 2 \}$) and $B = [1,4]$, find $f(A)$ and $f(B)$. Does $f(A\cap B) = f(A)\cap f(B)$ in this case? Does $f(A\cup B)=f(A) \cup f(B)$?
        \item Find two sets $A$ and $B$ for which $f(A\cap B) \neq f(A) \cap f(B)$.
        \item Show that, for an arbituary function $g: \mathbb{R} \rightarrow \mathbb{R}$, it is always true that $g(A \cap B) \subseteq g(A) \cap g(B)$ for all sets $A,B \subseteq \mathbb{R}$.
        \item Form and prove a conjecture about the relationship between $g(A \cup B)$ and $g(A) \cup g(B)$ for an arbituary function $g$.
    }

    \sol{
        \ea{
            \item For $f(x)=x^2$, $f(A)=f([0,2])=[0,4]$ and $f(B) = f([1,4]) = [1, 16]$. \begin{align*}
                f(A\cap B) = f([0,2]\cap [1, 4]) = f([1, 2]) = [1,&4] = [0, 4] \cap [1, 16] = f([1, 2]) \cap f([2, 4]) =f(A) \cap f(B) \\
                f(A \cup B) = f([0,2]\cup [1,4]) = f([0,4]) = [0,&16] = [0, 4] \cup [1, 16] = f([0,2]) \cup f([1, 4]) = f(A) \cup f(B)
            \end{align*}
            \item Consider $A = [0, 2]$ and $B = [-2, 0]$. $f(A \cap B) = \{0\}$, but $f(A) \cap f(B) = [0, 4]$.
            \item Suppose $y \in g(A \cap B)$, then $\exists x \in A \cap B$ such that $g(x) = y$. This implies that $x \in A$ and $x \in B$, so $x \in A \cap B$, hence $y \in g(A \cap B)$. Note that contrary may not always be true as it is possible for $x_1 \in A \backslash B$ and $x_2 \in B \backslash A$ such that $g(x_1) = g(x_2)$.
            \item I conjecture that $g(A \cup B) = g(A) \cup g(B)$. To prove this, we have to show inclusion both ways: \er{
                \item Let $y \in g(A\cup B)$, then $\exists x \in A \cup B$ such that $y=g(x)$. This implies that $x \in A$ or $x \in B$, so $y \in g(A)$ or $y \in g(B)$, hence $y \in g(A) \cup g(B)$.
                \item Let $y \in g(A) \cup g(B)$, then $y \in g(A)$ or $y \in g(B)$, implying $x \in A$ or $x \in B$ such that $y =g(x)$. So $x \in A \cup B$, hence $y \in g(A \cup B)$.
            }
        }
    }
}

\bx{%Exercise 1.2.8
    Here are two important definitions related to a function $f: A \rightarrow B$. The function $f$ is \textit{one-to-one} $(1-1)$ if $a_1 \neq a_2$ in $A$ implies that $f(a_1) \neq f(a_2)$ in $B$. The function $f$ is \textit{onto} if, given any $b \in B$, it is possible to find an element $a \in A$ for which $f(a) = b$
    Give an example of each or state that the request is impossible:
    \ea{
        \item $f: \mathbb{N} \rightarrow \mathbb{N}$ that is 1-1 but not onto.
        \item $f: \mathbb{N} \rightarrow \mathbb{N}$ that is onto but not 1-1.
        \item $f: \mathbb{N} \rightarrow \mathbb{Z}$ that is 1-1 and onto.
    }

    \sol{
        \ea{
            \item Let $f(x) = x+1$, which is 1-1 but does not have a solution to $f(x) = 1$, hence not onto.
            \item Let $f(x) = 1$ for $x=1$ and $f(x) = x-1$ for $x > 1$, which is onto but not 1-1 as $f(1) = f(2) = 1$.
            \item Let $f(x) = n/2$ when $n$ is even and $f(x) = - \frac{x-1}{2}$ when $n$ is odd.
        }
    }
}

\bx{%Exercise 1.2.9
    Given a function $f: D \rightarrow \mathbb{R}$ and a subset $B \subseteq \mathbb{R}$, let $f^{-1}(B)$ be the set of all points from the domain $D$ that get mapped into $B$; that is $f^{-1}(B) = \{x\in D:f(x) \in B\}$. This set is called the \textit{preimage} of $B$.
    \ea{
        \item Let $f(x) = x^2$. If $A$ is the closed interval $[0,4]$ and $B$ is the closed interval $[-1, 1]$, find $f^{-1}(A)$ and $f^{-1}(B)$. Does $f^{-1}(A\cap B) = f^{-1}(A) \cap f^{-1}(B)$ in this case? Does $f^{-1}(A \cup B) = f^{-1}(A) \cup f^{-1}(B)$?
        \item The good behaviour of preimages demonstrated in (a) is completely general. Show that for an arbituary function $g: \mathbb{R} \rightarrow \mathbb{R}$, it is always true that $g^{-1}(A \cap B) = g^{-1}(A) \cap g^{-1}(B)$ and $g^{-1}(A \cup B) = g^{-1}(A) \cup g^{-1}(B)$ for all sets $A,B \subseteq \mathbb{R}$.
    }

    \sol{
        \ea{
            \item For $f(x) = x^2$, $f^{-1}(A) = [-2, 2]$ and $f^{-1}(B)=[-1, 1]$. $f^{-1}(A \cap B) = f^{-1}([0, 1]) = [-1, 1] = f^{-1}(A) \cap f^{-1}(B)$. Similarly, $f^{-1}(A \cup B) = f^{-1}([-1, 4]) = [-2, 2] = f^{-1}(A) \cup f^{-1}(B)$.
            \item To show that $g^{-1}(A \cap B) = g^{-1}(A) \cap g^{-1}(B)$, we have to show inclusion both ways: \er{
                \item Let $x \in g^{-1}(A \cap B)$, so $g(x) \in A \cap B$, which implies $g(x) \in A$ and $g(x) \in B$. This shows that $x \in g^{-1}(A)$ and $x \in g^{-1}(B)$, hence $x \in g^{-1}(A) \cap g^{-1}(B)$.
                \item Let $x \in g^{-1}(A) \cap g^{-1}(B)$, so $x \in g^{-1}(A)$ and $x\in g^{-1}(B)$, then $g(x) \in A$ and $g(x) \in B$. This implies that $g(x) \in A \cap B$, so $x \in g^{-1}(A \cap B)$.
            }

            Showing $g^{-1}(A \cup B) = g^{-1}(A) \cup g^{-1}(B)$ is obvious using Exercise 1.2.7 (d).
        }
    }
}

\bx{ % Exercise 1.2.10
    Decide which of the following are true statements. Provide a short justification for those that are valid and a counterexample for those that are not:
    \ea{
        \item Two real numbers satisfy $a < b$ if and only if $a < b + \epsilon$ for every $\epsilon > 0$.
        \item Two real numbers satisfy $a < b$ if $a < b + \epsilon$ for every $\epsilon > 0$.
        \item Two real numbers satisfy $a \leq b$ if and only if $a < b + \epsilon$ for every $\epsilon > 0$.
    }
    \sol{
        \ea{
            \item False. Consider the case where $a<b+\epsilon$ is true but $a = b$.
            \item False. Same reasoning as above.
            \item True. Firsly suppose $a < b + \epsilon$ for all $\epsilon > 0$. We need to show this implies $a \leq b$. We either have $a \leq b$ or $a > b$. However, $a > b$ is not possible as this implies there exists an $\epsilon$ small enough such that $a > b+ \epsilon$.
            Secondly, suppose $a \leq b$. It is obvious that $a < b + \epsilon$ for all $\epsilon > 0$.
        }
    }
}

\bx{ % Exercise 1.2.11
    Form the logical negation of each claim. One trivial way to do this is to simply add "It is not the case that..." in front of each assertion. To make this more interesting, fashion the negation into a positive statement that avoids using the word "not" altogether. In each case, make an intuitive guess as to whether the claim or its negation is the true statement.
    \ea{
        \item For all real numbers satisfying $a < b$, there exists an $n \in \mathbb{N}$ such that $a + 1/n < b$.
        \item There exists a real number $x > 0$ such that $x < 1/n$ for all $n \in \mathbb{N}$.
        \item Between every two distinct real numbers there is a rational number.
    }
    \sol{
        \ea{
            \item For all $n \in \mathbb{N}$, there exists $a,b \in \mathbb{R}$ such that $a + 1/n < b$. [FALSE]
            \item For all real number $x > 0$, there exists an $n \in \mathbb{N}$ such that $x \geq 1/n$. [TRUE]
            \item There exists two real numbers $a < b$ such that if $r < b$ then $r < a$ for all $r \in \mathbb{Q}$. [FALSE]
        }
    }
}

\bx{ % Exercise 1.2.12
    Let $y_1=6$, and for each $n \in \mathbb{N}$ define $y_{n+1} = (2y_n-6)/3$.
    \ea{
        \item Use induction to prove that the sequence satisfies $y_n > -6$ for all $n \in \mathbb{N}$.
        \item Use another induction argument to show the sequence $(y_1, y_2, y_3, \dots)$ is decreasing.
    }
    \sol{
        \ea{
            \item For $n=1$, $y_1 = 6 > -6$ (Base Case). Suppose $y_n > -6$ for some $n \in \mathbb{N}$. \[
            y_{n+1}=\frac{2y_n-6}{3} > \frac{2(-6)-6}{3}=-6
            \]
            Hence, by induction, $y_n > -6$ for all $n \in \mathbb{N}$.
            \item Suppose $y_{n+1} < y_n$. The base case works as $y_2 = 2 < 6 =y_1$. Then, \begin{align*}
                y_{n+1} < y_n \implies& 2y_{n+1} - 6 < 2y_n -6 \\
                    \implies& \frac{2y_{n+1}-6}{3} < \frac{2y_n-6}{3} \\
                    \implies& y_{n+2} < y_{n+1}
            \end{align*}
            Thus, $y_{n+1}<y_n$ is true for all $n \in \mathbb{N}$.
        }
    }
}