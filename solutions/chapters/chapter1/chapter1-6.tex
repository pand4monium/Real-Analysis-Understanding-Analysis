\section{Cantor's Theorem}

\bx{ % Exercise 1.6.1
    Show that $(0,1)$ is uncountable if and only if $\mathbb{R}$ is uncountable.
    \sol{
        In Exercise 1.5.4 (a), we have found a bijection $f: (0,1) \rightarrow \mathbb{R}$. Suppose $g: (0,1) \rightarrow \mathbb{N}$ is some map, we must show $g$ is bijective if and only if $(g \circ f): \mathbb{R} \rightarrow \mathbb{N}$ is bijective. This is true as if $g$ is bijective, then $(g \circ f)$ is bijective as it is the composition of bijective functions. Similarly, if $(g \circ f)$ is bijective, then $(g \circ f) \circ f^{-1} = g$ is bijective.
        \medskip
        
        \noindent This proof works as $(0,1)$ is uncountable, implies that $g$ is not bijective, then $(g \circ f): \mathbb{R} \rightarrow \mathbb{N}$ is not bijective, implying that $\mathbb{R}$ is not countable and hence uncountable.
    }
}

\bx{ % Exercise 1.6.2
    \ea{
        \item Explain why the real number $x = .b_1b_2b_3b_4\dots$ cannot be $f(1)$.
        \item Now, explain why $x \neq f(2)$, and in general why $x \neq f(n)$ for any $n \in \mathbb{N}$.
        \item Point out the contradiction that arises from these observations and conclude that $(0,1)$ is uncountable.
    }
    \sol{\ea{
        \item Since $b_1 \neq a_{11}$, the first digit of $f(1)$ differ from $x$, hence cannot be equal.
        \item Since $b_n \neq a_{nn}$, the $n$-th digit of $f(n)$ differ from $x$, hence cannot be equal.
        \item Since $x$ is not in the list, it is a contradiction.
    }}
}

\bx{ % Exercise 1.6.3
    Supply rebuttals to the following complaints about the proof of Theorem 1.6.1. \ea{
        \item Every rational number has a decimal expansion, so we could apply this same argument to show that the set of rational numbers between 0 and 1 is uncountable. However, because we know that any subset of $\mathbb{Q}$ must be countable, the proof of Theorem 1.6.1 must be flawed.
        \item Some numbers have \textit{two} different decimal representations. Specifically, any decimal expansion that terminates can also be written with repeating 9's. For instance, 1/2 can be written as .5 or as $.4999\dots$. Doesn't this cause some problems?
    }

    \sol{
        \ea{
            \item False, since the constructed number has aninfinte number of decimals, it is irrational.
            \item No, since changing the $n$-th digit would still result in a different number.
        }
    }
}

\bx{ % Exercise 1.6.4
    Let $S$ be the set consisting of all sequences of 0's and 1's. Observe that $S$ is not a particular sequence, but rather a large set whose elements are sequences; namely, \[
        S = \{(a_1, a_2, a_3, \dots): a_n = 0 or 1\}.
    \] As an example, the sequence $(1,0,1,0,1,0,1,0,\dots)$ is an element of $S$, as is the sequence $(1,1,1,1,1,1,\dots)$.

    Give a rigorous argument showing $S$ is uncountable.
    \sol{
        Similar to Cantor's Theorem and the proof in Exercise 1.6.1, suppose there exists a function $f: \mathbb{N} \rightarrow S$ that is 1-1 and onto. For each $n \in \mathbb{N}$, $f(n)$ is an element in $S$, represented as: \[
            f(n) = (a_{n1}, a_{n2}, a_{n3}, \dots)
        \]where $a_{nm} = 0$ or $1$ for each $m,n\in \mathbb{N}$.
        We construct a $s \in S$, with $s = (b_1, b_2, b_3, \dots)$, where $b_n=0$ if $a_nn=1$ and $b_n=1$ otherwise. Since the $n$-th digit of sequence $s$ differs from $f(n)$ for all $n \in \mathbb{N}$, $s \notin S$, which is a contradiction. Hence $S$ is not countable, therefore uncountable.

        \bigskip
        Alternatively, we can define $g: S \rightarrow \bar{S}$ where $g(a_1, a_2, a_3, \dots) = .a_1a_2a_3\dots$ which is trivially bijective, and $h: \mathbb{R}_2 \rightarrow \mathbb{R}$ which converts a number in base 2 to base 10 and is clearly bijective. Hence $h \circ g: S \rightarrow \mathbb{R}$ is bijective and hence shows that $S \sim \mathbb{R}$ so $S$ is uncountable.
    }
}

\bx{ % Exercise 1.6.5
    \ea{
        \item Let $A=\{a,b,c\}$. List the eight elements of $P(A)$. (Do not forget that $\emptyset$ is considered to be a subset of every set.)
        \item If $A$ is finite with $n$ elements, show that $P(A)$ has $2^n$ elements.
    }
    \sol{\ea{
        \item $P(A) = \{\emptyset, \{a\}, \{b\}, \{c\}, \{a,b\}, \{a, c\}, \{b,c\}, \{a,b,c\}\}$
        \item There are $n$ elements, with $2$ possible states: included or excluded, implying that there are $2^n$ elements.
    }}
}