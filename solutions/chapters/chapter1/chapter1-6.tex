\section{Cantor's Theorem}

\bx{ % Exercise 1.6.1
    Show that $(0,1)$ is uncountable if and only if $\mathbb{R}$ is uncountable.
    \sol{
        In Exercise 1.5.4 (a), we have found a bijection $f: (0,1) \rightarrow \mathbb{R}$. Suppose $g: (0,1) \rightarrow \mathbb{N}$ is some map, we must show $g$ is bijective if and only if $(g \circ f): \mathbb{R} \rightarrow \mathbb{N}$ is bijective. This is true as if $g$ is bijective, then $(g \circ f)$ is bijective as it is the composition of bijective functions. Similarly, if $(g \circ f)$ is bijective, then $(g \circ f) \circ f^{-1} = g$ is bijective.
        \medskip
        
        \noindent This proof works as $(0,1)$ is uncountable, implies that $g$ is not bijective, then $(g \circ f): \mathbb{R} \rightarrow \mathbb{N}$ is not bijective, implying that $\mathbb{R}$ is not countable and hence uncountable.
    }
}

\bx{ % Exercise 1.6.2
    \ea{
        \item Explain why the real number $x = .b_1b_2b_3b_4\dots$ cannot be $f(1)$.
        \item Now, explain why $x \neq f(2)$, and in general why $x \neq f(n)$ for any $n \in \mathbb{N}$.
        \item Point out the contradiction that arises from these observations and conclude that $(0,1)$ is uncountable.
    }
    \sol{\ea{
        \item Since $b_1 \neq a_{11}$, the first digit of $f(1)$ differ from $x$, hence cannot be equal.
        \item Since $b_n \neq a_{nn}$, the $n$-th digit of $f(n)$ differ from $x$, hence cannot be equal.
        \item Since $x$ is not in the list, it is a contradiction.
    }}
}