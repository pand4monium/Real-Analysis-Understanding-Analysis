\section{Properties of Infinite Series}

\bx{ % 2.7.1
    Proving tha Alternating Series Test (Theorem 2.7.7) amounts to showing that the sequence of partial sums \[
    s_n = a_1 - a_2 +a_3 - \dots \pm a_n\] converges. (The opening example in Section 2.1 includes a typical illustration of $(s_n)$.) Different characterizations of completeness lead to different proofs. \ea{
        \item Prove the Alternating Series Test by showing that $(s_n)$ is a Cauchy sequence.
        \item Supply another proof for this result using the Nested Interval Property (Theorem 1.4.1).
        \item Consider the subsequences $(s_{2n})$ and $(s_{2n+1})$, and show how the Monotone Convergence Theorem leads to a third proof for the Alternating Series Test.
    }
    \sol{\ea{
        \item Let $N \in \mathbb{N}$ be even and let $n > N$. Since the series is alternating, we have \[
        s_N \leq s_n \leq s_{N+1}\] A similar statement can be made if $N$ is odd. Obviously, $|s_{N+1} - s_N| = |a_{N+1}|$ can be made as small as we want by increasing $N$. Hence we can set $N$ large enough so that $|a_N| < \epsilon/2$, giving \[
        |s_m - s_n| \leq |s_m - s_N| + |s_n - s_N| < \epsilon/2 + \epsilon/2 = \epsilon\] showing that $(s_n)$ is Cauchy, hence converges.

        \item Let $I_1 = [s_2, s_1]$ and in general, $I_n = [s_{2n}, s_{2n-1}]$. Since $a_n$ is decreasing and it is an alternating series, we get $I_{n+1} \subseteq I_n$. Considering the length of $I_n$, which equals to the length $|s_{2n} - s_{2n-1}| = a_{2n}$. If we set $N$ large enough, we can get $|a_{2n}| < \epsilon$. By NIP, there exists $x \in \bigcap^\infty_{n=1} I_n$. By order limit theorem, $\lim s_{2n} \leq x \leq \lim s_{2n-1}$, hence $\lim s_n = x$, showing that the series converges.
        \item We can regroup the elements in the partial sum. Consider \[
        s_{2n+1} = a_1 - (a_2 - a_3) - \dots - (a_{2n} - a_{2n+1}) \leq a_1\] Thus $s_{2n+1} \rightarrow s$ by MCT. To show this hold true for $(s_{2n})$, note that $s_{2n} = s{2n+1} - a_{2n+1}$ with $(a_{2n+1}) \rightarrow 0$, by triangle inequality \[
        |s_{2n} - s| \leq |s_{2n} - s{2n+1}| + |s_{2n+1} - s| < \epsilon/2 + \epsilon/2 = \epsilon\]And we are done.
    }}
}

\bx{ % 2.7.2
    Decide whether each of the following series converges or diverges: \ea{
        \item $\sum_{n=1}^{\infty} \frac{1}{2^n + n}$ 
        \item $\sum_{n=1}^{\infty} \frac{\sin (n)}{n^2}$
        \item $1 - \frac34 + \frac46 - \frac58 + \frac6{10} - \frac7{12} + \dots$
        \item $1 + \frac12 - \frac13 + \frac14 + \frac15 - \frac16 + \frac17 + \frac18 -\frac19 + \dots $
        \item $1 - \frac1{2^2} + \frac{1}{3} - \frac{1}{4^2} + \frac15 - \frac{1}{6^2} + \frac17 - \frac{1}{8^2}$
    }
    \sol{\ea{
        \item \[0 < \frac{1}{2^n+n} \leq \frac{1}{2^n} \implies 0 < \sum_{n=1}^{\infty} \frac{1}{2^n + n} < \sum_{n=1}^{\infty} \frac{1}{2^n} = 1\] Hence converges by comparison test.
        \item By Theorem 2.7.19, if $\sum_{n=1}^{\infty} \left| \frac{\sin (n)}{n^2} \right|$ converges, so does $\sum_{n=1}^{\infty} \frac{\sin (n) }{n^2}$. By comparison test, \[
        0 < \sum_{n=1}^{\infty} \left|\frac{\sin (n)}{n^2}\right| < \sum_{n=1}^{\infty} \frac{1}{n^2} = \frac{\pi^2}{6} \] hence the series converges.
        \item By comparison test, \[
        \frac12 \sum_{n=1}^{\infty} (-1)^{n+1} < \sum_{n=1}^{\infty} (-1)^n \frac{n+1}{2n}\] thus does diverging.
        \item Grouping terms give \[
        \frac1n + \left( \frac{1}{n+1} - \frac{1}{n+2} \right) \geq \frac1n\] By comparison test \[
        \sum_{n=1}^{\infty} \frac{1}{3n-2} + \frac{1}{3n-1} - \frac{1}{3n} > \sum_{n=1}^{\infty} \frac{1}{3n-2} > 1+ \sum_{n=2}^{\infty} \frac{1}{3n-3} = 1+\frac13 \sum_{n=1}^{\infty} \frac1n \] which diverges, hence the series diverges.
        \item Splitting the series gives\[
        s_{2n} = \sum_{k=1}^{\infty} \frac{1}{2k-1} + \sum_{k=1}^{\infty} \frac1{(2n)^2} > 1 + \frac12 \sum_{n=1}^{\infty} \frac1n + \frac{\pi^2}{24}\] which diverges by comparison test.
    }}
}

\bx{ % 2.7.3
    \ea{
        \item Provide the details for the proof of the Comparison Test (Theorem 2.7.4) using the Cauchy Criterion for Series.
        \item Give another proof for the Comparison Test, this time using the Monotone Convergence Theorem.
    }
    \sol{
        Suppose $a_n, b_n \geq 0$, $a_n \leq b_n$ and define $s_n = a_1 + \dots + a_n$, $t_n = b_1 + \dots + b_n$. \ea{
            \item We have $|a_m + \dots a_n| \leq |b_m + \dots + a_n| < \epsilon$, implying that $\sum_{n=1}^{\infty}a_n$ converges by Cauchy criterion. The other direction is similar, with $ |b_m + \dots + b_n | \geq |a_m + \dots + a_n | > \epsilon$, showing that $(s_n)$ diverges implies $(t_n)$ mst also diverge.
            \item Since $(t_n) \rightarrow t$. This implies that $s_n$ is bounded since $s_n \leq t_n < t$. With $a_n \geq 0$, MCT tells us that $(s_n)$ converges. The other direction can be proven by contradiction, assuming that $(t_n)$ is bounded tells us $(s_n)$ is bounded by MCT, which contradicts the fact that $(s_n)$ diverges.
        }
    }
}

\bx{ % 2.7.4
    Give an example of each or explain why the request is impossible, referencing the proper theorem(s). \ea{
        \item Two series $\sum x_n$ and $\sum y_n$ that both diverge but where $\sum x_ny_n$ converges.
        \item A convergent series $\sum x_n$ and a bounded sequence $(y_n)$ such taht $\sum x_ny_n$ diverges.
        \item Two sequences $(x_n)$ and $(y_n)$ where $\sum x_n$ and $\sum (x_n + y_n)$ both converge but $\sum y_n$ diverges.
        \item A sequence $(x_n)$ satisfying $0 \leq x_n \leq 1/n$ where $\sum (-1)^n x_n$ diverges.
    }
    \sol{\ea{
        \item Let $x_n = y_n = 1/n$, hence $\sum x_n = \sum y_n = \sum 1/n$ which diverges, but $\sum x_ny_n = \sum 1/n^2 = \pi^2 / 6$ which converges.
        \item Let $x_n = (-1)^n 1/n$, then $\sum x_n$ converges by Alternating Series Test, and $y_n = (-1)^n$ which is bounded. Hence $\sum x_ny_n = \sum 1/n$ which diverges.
        \item Impossible. By Algebraic Limit Theorem, $\sum y_n = \sum (x_n + y_n) - \sum x_n$ must converge.
        \item Let \[
        x_n = \begin{cases*}
            \frac1n &\text{ if } n \text{ is even} \\
            0 &\text{ otherwise}
        \end{cases*}\] then $\sum (-1)^n x_n = \sum 1/(2n)$ which diverges by Comparison Test with harmonic series.
    }}
}

\bx{ % 2.7.5
    Now that we have proved the basic facts about geometric series, supply a proof for Corollary 2.4.7.
    \sol{
        Corollary 2.4.7 states that the series $\sum_{n=1}^{\infty} 1/n^p$ \textit{converges if and only if } $p>1$.

        
        \noindent Using the Cauchy Condensation Test, the series $\sum_{n=1}^{\infty} 1/n^p$ converges if and only if the series \[
        \sum_{n=0}^{\infty} 2^n \frac{1}{(2^n)^p} = \sum_{n=0}^{\infty} 2^{n-np} = \sum_{n=0}^{\infty} \left(2^{1-p}\right)^n 
        \] converges. This is a geometric series that converges if and only if $2^{1-p} < 1 \implies p > 1$.
    }
}

\bx{ % 2.7.6
    Let's say that a series \textit{subverges} if the sequence of partial sums contains a subsequence that converges. Consider the (invented) definition for a moment, and then decide which of the following statements are valid propositions about subvergent series: \ea{
        \item If $(a_n)$ is bounded, then $\sum a_n$ subverges.
        \item All convergent series are subvergent.
        \item If $\sum |a_n|$ subverges, then $\sum a_n$ subverges as well.
        \item If $\sum a_n$ subverges, then $(a_n)$ has a convergent subsequence.
    }
    \sol{\ea{
        \item False. Let $a_n = 1$, then $\sum a_n = n$ which does not subverge.
        \item True. Every subsequence will converge to the same limit.
        \item True. Since $|a_n| \geq 0$, then $s_n = \sum_{i=1}^{n} |a_i|$ is increasing and contains a converging subsequence. Hence $s_n \leq M$ is bounded. By triangle inequality, $|\sum a_n| \leq \sum |a_n| = s_n \leq M$, then $\sum a_n$ is bounded. By BWT, we can find a convergent subsequence, thus subverging.
        \item False. Consider the sequence $(1, -1, 2, -2, \dots)$ which every even partial sum converges to zero, but the series does not converge.
    }}
}

\bx{ % 2.7.7
    \ea{
        \item Show that if $a_n > 0$ and $\lim (na_n) = l$ with $l \neq 0$, then the series $\sum a_n$ diverges.
        \item Assume $a_n > 0$ and $\lim (n^2a_n)$ exists. Show that $\sum a_n$ converges.
    }
    \sol{\ea{
        \item For $\epsilon > 0$, we get $na_n \in (l - \epsilon, l + \epsilon)$ implying $a_n > (l - \epsilon)(1/n)$. Using comparison text, we get $\sum a_n > (l-\epsilon) \sum 1/n$ which diverges, thus $\sum a_n$ diverges.
        \item Similarly, we get $n^2a_n \in (l-\epsilon, l+\epsilon)$ so $a_n < \frac{l+\epsilon}{n^2}$. By comparison test, $\sum a_n < (l + \epsilon)\sum \frac{1}{n^2}$, thus converging.
    }}
}

\bx{ %2.7.8
    Consider each of the following propositions. Provide short proofs for those that are true and counterexamples for any that are not. \ea{
        \item If $\sum a_n$ converges absolutely, then $\sum a_n^2$ also converges absolutely.
        \item If $\sum a_n$ converges and $(b_n)$ converges, then $\sum a_nb_n$ converges.
        \item If $\sum a_n$ converges conditionally, then $\sum n^2a_n$ diverges.
    }
    \sol{\ea{
        \item True. Since $\sum |a_n|$ converges, there exists $N>0$ such that for $n>N$, $0< a_n^2 < |a_n| < 1$. By comparison test, $\sum a_n^2$ also converges absolutely.
        \item False. Let $a_n = b_n = (-1)^n / \sqrt{n}$, with $\sum a_n$ converging by Alternating Series Test, but $\sum a_nb_n = \sum 1/n$ which diverges.
        \item True. Suppose $\sum n^2 a_n$ converges, since $(n^2a_n) \rightarrow 0$, we have $|n^2a_n| < \epsilon$ for $n > N$, implying $|a_n| < 1/n^2$. By comparison test, $\sum |a_n| < \sum 1/n^2$ converges, implying $\sum a_n$ converges absolutely, which is a contradiction. Therefore $\sum n^2 a_n$ must diverge.
    }}
}

\bx{ % 2.7.9
    \textbf{(Ratio Test)} Given a series $\sum_{n=1}^{\infty} a_n$ with $a_n \neq 0$, the Ratio Test states that if $(a_n)$ satisfies \[\lim \left|\frac{a_{n+1}}{a_n}\right| = r <1,\]then the series converges absolutely. \ea{
        \item Let $r'$ satisfy $r < r'<1$. Explain why there exists an $N$ such that $n \geq N$ implies $|a_{n+1}| \leq |a_n|r'$.
        \item Why does $|a_N| \sum(r')^n$ converge?
        \item Now, show that $\sum |a_n|$ converges, and conclude that $\sum a_n$ converges.
    }
    \sol{\ea{
        \item We are given \[
        \left|\left|\frac{a_{n+1}}{a_n}\right|-r\right| < \epsilon\] Since $1>r'>r$, we can set $\epsilon=r - r'$ meaning the neighbourhood \[
        \left|\frac{a_{n+1}}{a_n}\right| \in (r-\epsilon, r+\epsilon) = (2r-r', r')\]which is all less than $r'$ meaning \[
        \left|\frac{a_{n+1}}{a_n}\right| \leq r' \implies |a_{n+1}| \leq |a_{n}|r'\]

        \item With $|a_N|$ being a constant and $\sum (r')^n$ converging as it is a geometric series with $r' < 1$, then $|a_N|\sum (r')^n$ converges.
        \item To show that $\sum |a_n|$ converges, it is sufficient to show that $\sum_{k=N}^{\infty} |a_k|$ converges as $\sum_{k=1}^{N-1} |a_k|$ is the sum of finite terms, hence a constant.
        
        Let $N$ be large enough such that for $n>N$, we have $|a_{n+1}| \leq |a_n| r'$. Applying this multiple times gives us $|a_n| \leq (r')^{n-N} |a_N|$ which gives, \[
        |a_N| + |a_{N+1}| + \dots + |a_n| \leq |a_N| + r' |a_N| + \dots + (r')^{n-N} |a_N|\] Factoring gives us \[
        \sum_{k =N}^{n} |a_k| \leq |a_N| \sum_{k=0}^{n-1}(r')^k\] which converges as $n\rightarrow \infty$ by part (b). Hence implying that the sum converges absolutely.
    }}
}

\bx{ % 2.7.10
    \textbf{(Infinite Products)} Review Exercise 2.4.10 about infinite products and then answer the following questions: \ea{
        \item Does $\frac21 \cdot \frac32 \cdot\frac54\cdot\frac98\cdot\frac{17}{16}\dots$ converge?
        \item The infinite product $\frac12\cdot\frac34\cdot\frac56\cdot\frac78\cdot\frac{9}{10}\dots$ certainly converges. (Why?) Does it converge to zero?
        \item In 1655, John Wallis famously derived the formula \[
        \left(\frac{2\cdot2}{1\cdot3}\right) \left(\frac{4\cdot4}{3\cdot5}\right) \left(\frac{6\cdot6}{5\cdot7}\right) \left(\frac{8\cdot8}{7\cdot9}\right) \dots = \frac{\pi}{2}\]Show that the left side of this identity at least converges to something. (A complete proof of this result is taken up in Section 8.3)
    }
    \sol{\ea{
        \item Using Exercise 2.4.10, \[
        \frac21 \cdot \frac32 \cdot\frac54\cdot\frac98\cdot\frac{17}{16}\dots = \prod_{n=0}^{\infty} \frac{2^n + 1}{2^n} = \prod_{n=0}^\infty \left(1 + \frac{1}{2^n}\right)\]converges as $\sum 2^{-n}$ converges by geometric series.
        \item Each term is positive and less than one, causing the product to be monotone decreasing and bounded below by zero, converging by MCT.
        
        Let $p_n = \prod_{n=1}^{\infty} \frac{2n-1}{2n}$ and $(p_n) \rightarrow p$. Suppose $p \neq 0$, then $(1/p_n) \rightarrow p^{-1}$. \[
        p_n^{-1} = \prod_{n=1}^{\infty} \frac{2n}{2n-1} = \prod_{n=1}^{\infty} (1 + \frac{1}{2n-1})\] Since $\sum 1/(2n-1)$ diverges by comparison test with the harmonic series, $(1/p_n)$ diverges. This is a contradiction, showing that $p = 0$.
        \item Rewriting the terms as a partial fraction, \[
        \frac{(2n)(2n)}{(2n-1)(2n+1)} = \frac{4n^2}{4n^2 - 1} = 1+\frac{1}{4n^2 - 1}\] By Exercise 2.4.10, $\sum 1/(4n^2-1)$ converges by comparison test with $\sum 1/n^2$. Hence the product converges.
    }}
}

\bx{ % 2.7.11
    Find an example of two series $\sum a_n$ and $\sum b_n$ both of which diverge for which $\sum \min \{a_n, b_n\}$ converges. To make it more challenging, produce examples where $(a_n)$ and $(b_n)$ are strictly positive and decreasing.
    \sol{
        Let $m_n = \min \{a_n, b_n\}$. Intuitively, $(m_n)$ must take an infinite amount of $(a_n)$ and $(b_n)$ terms, as removing the finite terms would imply that either $\sum a_n$ or $\sum b_n$ would converge.

        We will begin by trying to obtain $m_n = 1/2^n$ to ensure that $\sum m_n$ converges. To have $\sum a_n$ and $\sum b_n$ diverging, we can simply repeat values when it is the greater value of the two.

        \begin{center}
            \begin{tabular}{ | c | c | c | c | c | c | c | c |}
                \hline
                $n$ & $1$ & $2$ & $3$ & $[4, 4+8 = 12)$ & $[12, 12+2^{11})$ & $[12 + 2^{11}, 12 + 2^{11} + 2^{12+2^11})$ & $\dots$\\
                \hline
                $m_n$ & $1/2$ & $1/4$ & $1/8$ & $1/2^n$ & $1/2^n$ & $1/2^n$ & $\dots$\\
                \hline
                $\min \{a_n, b_n\}$ & $a_n$ & $b_n$ & $b_n$ & $a_n$ & $b_n$ & $a_n$ & $\dots$ \\
                \hline
                $a_n$ & $1/2$ & $1/2$ & $1/2$ & $1/2^n$ & $1/2^{11}$ & $1/2^n$ & $\dots$ \\
                \hline
                $b_n$ & $1$ & $1/4$ & $1/8$ & $1/8$ & $1/2^n$ & $1/2^{12 + 2^{11}}$ & $\dots$ \\
                \hline
            \end{tabular}
        \end{center}

        Now we show that this construction satisfies the conditions. It is obvious that $\sum \min\{a_n, b_n\}$ converges as stated above. For $\sum a_n$ and $\sum b_n$, we have to group the terms as shown above. Let $k_1 = 1$, $k_n = k_{n-1} + 2^{k_{n-1}-1}$ Then \[
        a_n = \begin{cases}
            1/2^n & \text{if } n \in [k_{2p-1}, k_{2p}) \\
            1/2^{k_{2p} -1} & \text{if } n \in [k_{2p} , k_{2p+1})
        \end{cases} \text{ and } b_n = \begin{cases}
            1/2^n & \text{if } n \in [k_{2p}, k_{2p+1}) \\
            1/2^{k_{2p-1} -1} & \text{if } n \in [k_{2p-1} , k_{2p})
        \end{cases}
        \]Hence both $\sum a_n$ and $\sum b_n$ diverges as the first case has its blocks sum to one, causing the sum to approach infinity.
    }
}

\bx{ % 2.7.12
    \textbf{(Summation-by-parts)} Let $(x_n)$ and $(y_n)$ by sequences, let $s_n = x_1 + x_2 + \dots + x_n$ and set $s_0 = 0$. Use the observation that $x_j = s_j - s_{j-1}$ to verify the formula \[
    \sum_{j=m}^{n}x_jy_j = s_ny_{n+1} - s_{m-1}y_m + \sum_{j=m}^{n} s_j(y_j - y_{j+1})\]
    \sol{
        \begin{align*}
            \sum_{j=m}^{n} x_jy_j & = \sum_{j=m}^{n} (s_j - s_{j-1})y_j \\
            & = \sum_{j=m}^{n} s_jy_j - \sum_{j=m}^{n} s_{j-1}y_j \\
            & = \sum_{j=m}^n s_jy_j - \sum_{j=m+1}^{n+1} s_jy_{j+1} \\
            & = \sum_{j=m}^n s_jy_j - \sum_{j=m}^n s_jy_{j+1} - s_{m-1}y_{m} + s_ny_{n+1} \\
            & = s_ny_{n+1} - s_{m-1}y_m + \sum_{j=m}^{n} s_j(y_j - y_{j+1})
        \end{align*}
    }
}

\bx{ % 2.7.13
    \textbf{(Abel's Test)} Abel's Test for convergence states that if the series $\sum_{k=1}^{\infty} x_k$ converges, and if $(y_k)$ is a sequence satisfying \[
    y_1 \geq y_2 \geq y_3 \geq \dots \geq 0\]then the series $\sum_{k=1}^{\infty} x_ky_k$ converges. \ea{
        \item Use Exercise 2.7.12 to show that \[
        \sum_{k=1}^{n} x_ky_k = s_ny_{n+1} + \sum_{k=1}^{n} s_k(y_k-y_{k+1})\]where $s_n = x_1 + x_2 + \dots x_n$.
        \item Use the Comparison Test to argue that $\sum_{k=1}^{\infty} s_n (y_k - y_{k+1})$ converges absolutely, and show how this leads directly to a proof of Abel's Test.
    }
    \sol{\ea{
        \item From Exercise 2.7.12,\begin{align*}
            \sum_{k=1}^{n}x_ky_k &= s_ny_{n+1} - s_{0}y_1 + \sum_{k=1}^{n} s_j(y_k - y_{k+1}) \\
            &= s_ny_{n+1} + \sum_{k=1}^{n} s_j(y_k - y_{k+1})
        \end{align*}
        \item Since $\sum x_k$ converges, we can bound $s_n \leq M$. By Comparison Test, we get \begin{align*}
        \sum_{k=1}^{\infty} |s_n (y_k - y_{k+1})| & \leq |M| \sum_{k=1}^{\infty} (y_k - y_{k+1}) \\
        & \leq |M| (y_1 - \lim y_n) = |M|(y_1-y)
        \end{align*}where $(y_n) \rightarrow y$. 

        Since both $s_ny_{n+1}$ and $\sum s_k (y_k - y_{k+1})$ converge, then $\sum_{k=1}^{\infty} x_ky_k$ also converge by algebraic limit theorem.
    }}
}

\bx{ % 2.7.14
    \textbf{(Dirichlet's Test)} Dirichlet's Test for convergence states that if the partial sums of $\sum_{k=1}^{\infty}x_k$ are bounded (but not necessarily convergent), and if $(y_k)$ is a sequence satisfying $y_1 \geq y_2 \geq y_3 \geq \dots \geq 0$ with $\lim y_k = 0$, then the series $\sum_{k=1}^{\infty} x_ky_k$ converges. \ea{
        \item Point out how the hypothesis of Dirichlet's Test differs from that of Abel's Test in Exercise 2.7.13, but show that essentially the same strategy can be used to provide a proof.
        \item Show how the Alternating Series Test (Theorem 2.7.7) can be derived as a special case of Dirichlet's Test.
    }
    \sol{\ea{
        \item Dirichlet's Test has the $(y_n)$ being convergent, while Abel's Test has $\sum x_n$ being convergent.
        
        To recall, Abel's Test uses the convergence of $\sum x_n$, to bound it. In addition, MCT tells us that $(y_n)$ converges as it is decreasing and bounded above. These two criteria help show that $\sum x_ky_k$ converges. 

        The conditions of Dirichlet's Test also lead to the same to criteria. Firstly, it is given that $\sum x_k$ is bounded. Secondly, it is given that $(y_k)$ converges to zero. Hence the same strategy can be used to provide a proof, thus left as an exercise to the reader.

        \item Letting $x_n = (-1)^n$ results in $\sum x_ky_k = \sum (-1)^k y_k$ converging, which satisfies the conditions of Alternating Series Test.
    }}
}