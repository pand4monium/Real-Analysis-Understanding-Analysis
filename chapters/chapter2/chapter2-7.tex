\section{Properties of Infinite Series}

\bx{ % 2.7.1
    Proving tha Alternating Series Test (Theorem 2.7.7) amounts to showing that the sequence of partial sums \[
    s_n = a_1 - a_2 +a_3 - \dots \pm a_n\] converges. (The opening example in Section 2.1 includes a typical illustration of $(s_n)$.) Different characterizations of completeness lead to different proofs. \ea{
        \item Prove the Alternating Series Test by showing that $(s_n)$ is a Cauchy sequence.
        \item Supply another proof for this result using the Nested Interval Property (Theorem 1.4.1).
        \item Consider the subsequences $(s_{2n})$ and $(s_{2n+1})$, and show how the Monotone Convergence Theorem leads to a third proof for the Alternating Series Test.
    }
    \sol{\ea{
        \item Let $N \in \mathbb{N}$ be even and let $n > N$. Since the series is alternating, we have \[
        s_N \leq s_n \leq s_{N+1}\] A similar statement can be made if $N$ is odd. Obviously, $|s_{N+1} - s_N| = |a_{N+1}|$ can be made as small as we want by increasing $N$. Hence we can set $N$ large enough so that $|a_N| < \epsilon/2$, giving \[
        |s_m - s_n| \leq |s_m - s_N| + |s_n - s_N| < \epsilon/2 + \epsilon/2 = \epsilon\] showing that $(s_n)$ is Cauchy, hence converges.

        \item Let $I_1 = [s_2, s_1]$ and in general, $I_n = [s_{2n}, s_{2n-1}]$. Since $a_n$ is decreasing and it is an alternating series, we get $I_{n+1} \subseteq I_n$. Considering the length of $I_n$, which equals to the length $|s_{2n} - s_{2n-1}| = a_{2n}$. If we set $N$ large enough, we can get $|a_{2n}| < \epsilon$. By NIP, there exists $x \in \bigcap^\infty_{n=1} I_n$. By order limit theorem, $\lim s_{2n} \leq x \leq \lim s_{2n-1}$, hence $\lim s_n = x$, showing that the series converges.
        \item We can regroup the elements in the partial sum. Consider \[
        s_{2n+1} = a_1 - (a_2 - a_3) - \dots - (a_{2n} - a_{2n+1}) \leq a_1\] Thus $s_{2n+1} \rightarrow s$ by MCT. To show this hold true for $(s_{2n})$, note that $s_{2n} = s{2n+1} - a_{2n+1}$ with $(a_{2n+1}) \rightarrow 0$, by triangle inequality \[
        |s_{2n} - s| \leq |s_{2n} - s{2n+1}| + |s_{2n+1} - s| < \epsilon/2 + \epsilon/2 = \epsilon\]And we are done.
    }}
}

\bx{ % 2.7.2
    Decide whether each of the following series converges or diverges: \ea{
        \item $\sum_{n=1}^{\infty} \frac{1}{2^n + n}$ 
        \item $\sum_{n=1}^{\infty} \frac{\sin (n)}{n^2}$
        \item $1 - \frac34 + \frac46 - \frac58 + \frac6{10} - \frac7{12} + \dots$
        \item $1 + \frac12 - \frac13 + \frac14 + \frac15 - \frac16 + \frac17 + \frac18 -\frac19 + \dots $
        \item $1 - \frac1{2^2} + \frac{1}{3} - \frac{1}{4^2} + \frac15 - \frac{1}{6^2} + \frac17 - \frac{1}{8^2}$
    }
    \sol{\ea{
        \item \[0 < \frac{1}{2^n+n} \leq \frac{1}{2^n} \implies 0 < \sum_{n=1}^{\infty} \frac{1}{2^n + n} < \sum_{n=1}^{\infty} \frac{1}{2^n} = 1\] Hence converges by comparison test.
        \item By Theorem 2.7.19, if $\sum_{n=1}^{\infty} \left| \frac{\sin (n)}{n^2} \right|$ converges, so does $\sum_{n=1}^{\infty} \frac{\sin (n) }{n^2}$. By comparison test, \[
        0 < \sum_{n=1}^{\infty} \left|\frac{\sin (n)}{n^2}\right| < \sum_{n=1}^{\infty} \frac{1}{n^2} = \frac{\pi^2}{6} \] hence the series converges.
        \item By comparison test, \[
        \frac12 \sum_{n=1}^{\infty} (-1)^{n+1} < \sum_{n=1}^{\infty} (-1)^n \frac{n+1}{2n}\] thus does diverging.
        \item Grouping terms give \[
        \frac1n + \left( \frac{1}{n+1} - \frac{1}{n+2} \right) \geq \frac1n\] By comparison test \[
        \sum_{n=1}^{\infty} \frac{1}{3n-2} + \frac{1}{3n-1} - \frac{1}{3n} > \sum_{n=1}^{\infty} \frac{1}{3n-2} > 1+ \sum_{n=2}^{\infty} \frac{1}{3n-3} = 1+\frac13 \sum_{n=1}^{\infty} \frac1n \] which diverges, hence the series diverges.
        \item Splitting the series gives\[
        s_{2n} = \sum_{k=1}^{\infty} \frac{1}{2k-1} + \sum_{k=1}^{\infty} \frac1{(2n)^2} > 1 + \frac12 \sum_{n=1}^{\infty} \frac1n + \frac{\pi^2}{24}\] which diverges by comparison test.
    }}
}

\bx{ % 2.7.3
    \ea{
        \item Provide the details for the proof of the Comparison Test (Theorem 2.7.4) using the Cauchy Criterion for Series.
        \item Give another proof for the Comparison Test, this time using the Monotone Convergence Theorem.
    }
    \sol{
        Suppose $a_n, b_n \geq 0$, $a_n \leq b_n$ and define $s_n = a_1 + \dots + a_n$, $t_n = b_1 + \dots + b_n$. \ea{
            \item We have $|a_m + \dots a_n| \leq |b_m + \dots + a_n| < \epsilon$, implying that $\sum_{n=1}^{\infty}a_n$ converges by Cauchy criterion. The other direction is similar, with $ |b_m + \dots + b_n | \geq |a_m + \dots + a_n | > \epsilon$, showing that $(s_n)$ diverges implies $(t_n)$ mst also diverge.
            \item Since $(t_n) \rightarrow t$. This implies that $s_n$ is bounded since $s_n \leq t_n < t$. With $a_n \geq 0$, MCT tells us that $(s_n)$ converges. The other direction can be proven by contradiction, assuming that $(t_n)$ is bounded tells us $(s_n)$ is bounded by MCT, which contradicts the fact that $(s_n)$ diverges.
        }
    }
}

\bx{ % 2.7.4
    Give an example of each or explain why the request is impossible, referencing the proper theorem(s). \ea{
        \item Two series $\sum x_n$ and $\sum y_n$ that both diverge but where $\sum x_ny_n$ converges.
        \item A convergent series $\sum x_n$ and a bounded sequence $(y_n)$ such taht $\sum x_ny_n$ diverges.
        \item Two sequences $(x_n)$ and $(y_n)$ where $\sum x_n$ and $\sum (x_n + y_n)$ both converge but $\sum y_n$ diverges.
        \item A sequence $(x_n)$ satisfying $0 \leq x_n \leq 1/n$ where $\sum (-1)^n x_n$ diverges.
    }
    \sol{\ea{
        \item Let $x_n = y_n = 1/n$, hence $\sum x_n = \sum y_n = \sum 1/n$ which diverges, but $\sum x_ny_n = \sum 1/n^2 = \pi^2 / 6$ which converges.
        \item Let $x_n = (-1)^n 1/n$, then $\sum x_n$ converges by Alternating Series Test, and $y_n = (-1)^n$ which is bounded. Hence $\sum x_ny_n = \sum 1/n$ which diverges.
        \item Impossible. By Algebraic Limit Theorem, $\sum y_n = \sum (x_n + y_n) - \sum x_n$ must converge.
        \item Let \[
        x_n = \begin{cases*}
            \frac1n &\text{ if } n \text{ is even} \\
            0 &\text{ otherwise}
        \end{cases*}\] then $\sum (-1)^n x_n = \sum 1/(2n)$ which diverges by Comparison Test with harmonic series.
    }}
}

\bx{ % 2.7.5
    Now that we have proved the basic facts about geometric series, supply a proof for Corollary 2.4.7.
    \sol{
        Corollary 2.4.7 states that the series $\sum_{n=1}^{\infty} 1/n^p$ \textit{converges if and only if } $p>1$.

        
        \noindent Using the Cauchy Condensation Test, the series $\sum_{n=1}^{\infty} 1/n^p$ converges if and only if the series \[
        \sum_{n=0}^{\infty} 2^n \frac{1}{(2^n)^p} = \sum_{n=0}^{\infty} 2^{n-np} = \sum_{n=0}^{\infty} \left(2^{1-p}\right)^n 
        \] converges. This is a geometric series that converges if and only if $2^{1-p} < 1 \implies p > 1$.
    }
}

\bx{ % 2.7.6
    Let's say that a series \textit{subverges} if the sequence of partial sums contains a subsequence that converges. Consider the (invented) definition for a moment, and then decide which of the following statements are valid propositions about subvergent series: \ea{
        \item If $(a_n)$ is bounded, then $\sum a_n$ subverges.
        \item All convergent series are subvergent.
        \item If $\sum |a_n|$ subverges, then $\sum a_n$ subverges as well.
        \item If $\sum a_n$ subverges, then $(a_n)$ has a convergent subsequence.
    }
    \sol{\ea{
        \item False. Let $a_n = 1$, then $\sum a_n = n$ which does not subverge.
        \item True. Every subsequence will converge to the same limit.
        \item True. Since $|a_n| \geq 0$, then $s_n = \sum_{i=1}^{n} |a_i|$ is increasing and contains a converging subsequence. Hence $s_n \leq M$ is bounded. By triangle inequality, $|\sum a_n| \leq \sum |a_n| = s_n \leq M$, then $\sum a_n$ is bounded. By BWT, we can find a convergent subsequence, thus subverging.
        \item False. Consider the sequence $(1, -1, 2, -2, \dots)$ which every even partial sum converges to zero, but the series does not converge.
    }}
}

\bx{ % 2.7.7
    \ea{
        \item Show that if $a_n > 0$ and $\lim (na_n) = l$ with $l \neq 0$, then the series $\sum a_n$ diverges.
        \item Assume $a_n > 0$ and $\lim (n^2a_n)$ exists. Show that $\sum a_n$ converges.
    }
    \sol{\ea{
        \item For $\epsilon > 0$, we get $na_n \in (l - \epsilon, l + \epsilon)$ implying $a_n > (l - \epsilon)(1/n)$. Using comparison text, we get $\sum a_n > (l-\epsilon) \sum 1/n$ which diverges, thus $\sum a_n$ diverges.
        \item Similarly, we get $n^2a_n \in (l-\epsilon, l+\epsilon)$ so $a_n < \frac{l+\epsilon}{n^2}$. By comparison test, $\sum a_n < (l + \epsilon)\sum \frac{1}{n^2}$, thus converging.
    }}
}

\bx{ %2.7.8
    Consider each of the following propositions. Provide short proofs for those that are true and counterexamples for any that are not. \ea{
        \item If $\sum a_n$ converges absolutely, then $\sum a_n^2$ also converges absolutely.
        \item If $\sum a_n$ converges and $(b_n)$ converges, then $\sum a_nb_n$ converges.
        \item If $\sum a_n$ converges conditionally, then $\sum n^2a_n$ diverges.
    }
    \sol{\ea{
        \item True. Since $\sum |a_n|$ converges, there exists $N>0$ such that for $n>N$, $0< a_n^2 < |a_n| < 1$. By comparison test, $\sum a_n^2$ also converges absolutely.
        \item False. Let $a_n = b_n = (-1)^n / \sqrt{n}$, with $\sum a_n$ converging by Alternating Series Test, but $\sum a_nb_n = \sum 1/n$ which diverges.
        \item True. Suppose $\sum n^2 a_n$ converges, since $(n^2a_n) \rightarrow 0$, we have $|n^2a_n| < \epsilon$ for $n > N$, implying $|a_n| < 1/n^2$. By comparison test, $\sum |a_n| < \sum 1/n^2$ converges, implying $\sum a_n$ converges absolutely, which is a contradiction. Therefore $\sum n^2 a_n$ must diverge.
    }}
}