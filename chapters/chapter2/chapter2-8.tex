\section{Double Summations and Products of Infinite Series}

\bx{ % 2.8.1
    Using tha particular array $(a_{ij})$ from Section 2.1, compute $\lim_{n \rightarrow \infty} s_{nn}$. How does this compare to the two iterated values for the sum already computed?
    \sol{
        To recall, the particular array in Section 2.1 is \[\begin{bmatrix}
            -1 & \frac12 & \frac14 & \frac18 & \frac1{16} & \dots \\[4pt] 
            0 & -1 & \frac12 & \frac14 & \frac18 & \dots \\[4pt]
            0 & 0 & -1 & \frac12 & \frac14 & \dots \\[4pt]
            0 & 0 & 0 & -1 & \frac12 & \dots \\[4pt]
            0 & 0 & 0 & 0 & -1 & \dots \\
            \vdots & \vdots & \vdots & \vdots & \vdots & \ddots \\
        \end{bmatrix}\] Doing some calculations, $s_{nn} = -2 + 1/2^{n-1}$, so $\lim s_{nn} = -2$. This is the same as when fixing $j$ and summing down each column, since each column has finitely many non-zero elements.
    }
}

\bx{ % 2.8.2
    Show that if the iterated series \[
    \sum_{i=1}^{\infty} \sum_{j=1}^{\infty} |a_{ij}|\]converges (meaning that for each fixed $i \in \mathbb{N}$ the series $\sum_{j=1}^{\infty} |a_{ij}|$ converges to some real number $b_i$, and the series $\sum_{i=1}^{\infty} b_i$ converges as well), then the iterated series \[
    \sum_{i=1}^{\infty} \sum_{j=1}^{\infty} a_{ij}\]converges. 
    \sol{
        This is obvious. We can use triangle inequality on the countable infinite terms in the double summation. To be more rigourous, let $c_i = \sum_{j=1}^\infty a_{ij}$. Since $\sum_{j=1}^\infty |a_{ij}|$ converges, $c_o$ converges as well. With $|c_i| \leq |b_i|$, then $\sum_{i=1}^{\infty} c_i$ converges by comparison text with $\sum_{i=1}^\infty |b_i|$. 
    }
}

\bx{ % 2.8.3
    \ea{
        \item Prove that $(t_{nn})$ converges.
        \item Now, use the fact that $(t_{nn})$ is a Cauchy Sequence to argue that $(s_{nn})$ converges.
    }
    \sol{\ea{
        \item Since $|a_{ij}| > 0$, $t_{nn}$ is monotone increasing. Hence for MCT to tell us it converges, we just need to show that the sequence is bounded.\[
        \sum_{i=1}^{\infty} \sum_{j=1}^{\infty} |a_{ij}| \geq \sum_{i=1}^{\infty} \sum_{j=1}^{n} |a_{ij}| \geq \sum_{i=1}^{n} \sum_{j=1}^n |a_{ij}| = t_{nn}\]
        
        \item Since $(t_{nn})$ is a Cauchy Sequence, for any $\epsilon > 0$ there exists $N$ such that if $p>q>N$, \begin{align*}
            \epsilon > |t_{pp} - t_{qq}| & = \left|\sum_{i=1}^{p} \sum_{j=1}^{p} |a_{ij}| - \sum_{i=1}^{q} \sum_{j=1}^{q} |a_{ij}| \right| \\
            & = \left| \sum_{i=q+1}^{p} \sum_{j=1}^{p} |a_{ij}| + \sum_{i=1}^{q} \sum_{j=q+1}^{p} |a_{ij}| + \sum_{i=1}^q \sum_{j=1}^{q} |a_{ij}| - \sum_{i=1}^{q} \sum_{j=1}^q  |a_{ij}| \right| \\
            & = \left| \sum_{i=q+1}^{p} \sum_{j=1}^{p} |a_{ij}| + \sum_{i=1}^{q} \sum_{j=q+1}^{p} |a_{ij}| \right| \\
            & \geq \left| \sum_{i=q+1}^{p} \sum_{j=1}^{p} a_{ij} + \sum_{i=1}^{q} \sum_{j=q+1}^{p} a_{ij} \right| \\
            & = \left| \sum_{i=q+1}^{p} \sum_{j=1}^{p} a_{ij} + \sum_{i=1}^{q} \sum_{j=q+1}^{p} a_{ij} + \sum_{i=1}^q \sum_{j=1}^{q} a_{ij} - \sum_{i=1}^{q} \sum_{j=1}^q  a_{ij} \right| \\
            & = \left|\sum_{i=1}^{p} \sum_{j=1}^{p} a_{ij} - \sum_{i=1}^{q} \sum_{j=1}^{q} a_{ij} \right| \\
            & = |s_{pp} - s_{qq}|
        \end{align*}thus $s_{nn}$ is also a Cauchy Sequence, hence converging.
    }}
}

\bx{ % 2.8.4
    \ea{
        \item Let $\epsilon > 0$ be arbitrary and argue that there exists an $N_1 \in \mathbb{N}$ such that $m,n \geq N_1$ implies $B - \frac{\epsilon}{2} < t_{mn} \leq B$.
        \item Now, show that there exists an $N$ such that \[
        |s_{mn} - S| < \epsilon\]for all $m,n \geq N$.
    }
    \sol{\ea{
        \item With $B$ being the upper bound on $\{t_{mn} : m,n\in \mathbb{N}\}$, we get $t_{mn} \leq B$. Lemma 1.3.8 tells us that there exists some $p,q$ such that $t_{pq} > B - \epsilon/2$. And since $p_1 \leq p_2$ and $q_1 \leq q_2 \implies t_{p_1q_1} \leq t_{p_2q_2}$, we can choose $N_1 = \max \{p,q\}$.
        \item By triangle inequality, $|s_{mn} - S| \leq |s_{mn} - s{nn}| + |s_{nn} - S|$. WLOG, let $m > n$, \[
        |s_{mn} - s{nn}| = \left|\sum_{i = n}^{m} \sum_{j=1}^{n} a_{ij}\right| \leq \sum_{i=n}^{m} \sum_{j-1}^{n} |a_{ij}| = |t_{mn} - t_{nn}| \leq \frac\epsilon2 \]from part (a), as long as $m,n \geq N_1$. Since $S= \lim s_{nn}$ there exists $N_2$ such that for $n \geq N_2$, $|s_{nn} - S| < \epsilon/2$, thus picking $N = \max \{N_1, N_2\}$ satisfies that condition.
    }}
}

\bx{ % 2.8.5
    \ea{
        \item Show that for all $m \geq N$ \[
        |(r_1 + r_2 + \dots + r_m) - S| \leq \epsilon\]Conclude that the iterated sum $\sum_{i=1}^{\infty} \sum_{j=1}^{\infty} a_{ij}$ converges to $S$.
        \item Finish the proof by showing that the other iterated sum, $\sum_{j=1}^{\infty} \sum_{i=1}^{\infty} a_{ij}$, converges to $S$ as well. Notice the same argument can be used once it is established that, for each fixed column $j$, the sum $\sum_{i=1}^{\infty} a_{ij}$ converges to some real number $c_j$.
    }
    \sol{\ea{
        \item Suppose $\sum_{j=1}^{\infty} a_{ij}$ converges tosome value $r_i$. We can choose $N_3$ large enough such that when $n> N_3$ and $k \leq m$, we get \[
        \left|r_k - \sum_{j=1}^{n} a_{kj}\right| \leq \frac{\epsilon}{2m}\] Then there must exists some $N > N_3$ such that when $m,n \geq N$, \begin{align*}
            \left| \sum_{i=1}^{m}r_i - S\right| & \leq \left| \sum_{i=1}^{m} - \sum_{i=1}^{m} r_i \sum_{j=1}^{n} a_{ij}\right| + \left| \sum_{i=1}^{m} \sum_{j=1}^{n} a_{ij} - S \right| \\
            & = \left| \sum_{i=1}^{m} \left( r_i - \sum_{j=1}^{n} a_{ij} \right) \right| + |s_{mn} - S| \\
            & \leq \sum_{i=1}^{m} \left| r_i - \sum_{j=1}^{n} a_{ij} \right| + | s_{mn} - S| \\
            & < \sum_{i=1}^{m} \left( \frac{\epsilon}{2m} \right) + \frac{\epsilon}{2} = \epsilon
        \end{align*} and thus $\sum_{i=1}^{\infty} \sum_{j=1}^{\infty} a_{ij}$ converges to $S$.

        \item To show that $\sum_{j=1}^{\infty} \sum_{i=1}^{\infty} a_{ij}$ converges to the same value $S$, we will use the same arguement. Our hypothesis guarentees that for each fixed column $j$, the series $\sum_{i=1}^{\infty} a_{ij}$ converges absolutely to some real number $c_j$. Using the same arguement as in (a), we can show that \[
        \left| (c_1 + c_2 + \dots + c_m) - S\right| \leq \epsilon\] for all $m \geq N$. Hence, the iterated sum $\sum_{j=1}^{\infty} \sum_{i=1}^{\infty} a_{ij}$ converges to the same value of $S$.
    }}
}