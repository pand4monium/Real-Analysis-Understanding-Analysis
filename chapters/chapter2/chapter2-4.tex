\section{The Monotone Convergence Theorem and a First Look at Infinite Series}

\bx{ % 2.4.1
    \ea{
        \item Prove that the sequence defined by $x_1 = 3$ and \[
        x_{n+1} = \frac{1}{a-x_n}\] converges.
        \item Now that we know $\lim x_n$ exists, explain why $\lim x_{n+1}$ must also exist and equal the same value.
        \item Take the limit of each side of the recursive equation in part (a) to explicitly compute $\lim x_n$.
    }
    \sol{\ea{
        \item Calculating the first few terms, we get $(3, 1, 1/3, \dots)$. I conjecture that \seq{x}{n} is decreasing. 
        \medskip

        Suppose $x_{n+1} < x_n$, then \[
        4 - x_n > 4 - x_{n+1} \implies \frac{1}{4-x_n} < \frac{1}{4-x_{n+1}} \implies x_{n+2} < x_{n+1}
        \]Thus, \seq{x}{n} is decreasing. Since $x_n < 4$ for all $n \in \mathbb{N}$, $4 - x_n > 0$ and hence $x_{n+1} > 0$. Since \seq{x}{n} is decreasing and bounded below, \seq{x}{n} converges.

        \item By definition of limits, changing the index of the sequence does not change its limit.
        \item Since $x = \lim x_n = \lim x_{n+1}$, \[
        x_{n+1} = \frac{1}{4 - x_n} \implies x = \frac{1}{4-x} \implies x^2 - 4x + 1 = 0 \implies x = 2 \pm \sqrt3 
        \] Since $x_n < 3$, then $x < 3$, hence $x \neq 2 +\sqrt3$ and $x = 2 - \sqrt3$.
    }}
}

\bx{ % 2.4.2
    \ea{
        \item Consider the recursively defined sequence $y_1 = 1$, \[
        y_{n+1} = 3-y_n,\] and set $y = \lim y_n$. Because \seq{y}{n} and \seq{y}{n+1} have the same limit, taking the limit across the recursive equation gives $y = 3 -y$. Solving for $y$, we conclude $\lim y_n = 3/2$.

        
        \noindent What is wrong with this argument?

        \item This time set $y_1 = 1$ and $y_{n+1} = 3 - \frac{1}{y_n}$. Can the strategy in (a) be applied to compute the limit of this sequence?
    }
    \sol{\ea{
        \item The argument only works if the series converges, which does not in this case.
        \item Yes it can as $0 < y_n < 3$ and \seq{y}{n} is increasing (Proving this is left as an exercise for the reader).
    }}
}

\bx{ % 2.4.3
    \ea{ 
        \item Show that \[
        \sqrt2, \sqrt{2+\sqrt2}, \sqrt{2+\sqrt{2+\sqrt2}}, \dots\] converges and find the limit.
        \item Does the sequence \[
        \sqrt2, \sqrt{2\sqrt2}, \sqrt{2\sqrt{2\sqrt2}}, \dots\] converge? If so, find the limit.
    }

    \sol{\ea{
        \item Let $x_1 = \sqrt2$ and $x_{n+1} = \sqrt{2 + x_n}$, clearly $x_2 > x_1$. Suppose $x_{n+1} > x_n$, then \[
        2 + x_{n+1} > 2 + x_n \implies \sqrt{2 + x_{n+1}} > \sqrt{2 + x_n} \implies x_{n+2} > x_{n+1}\] showing that $x_n$ is monotonically increasing.
        \medskip

        \noindent Next we show that $x_n \leq 2$, clearing $x_1 = \sqrt2 \leq 2$. Suppose $x_n \leq 2$, then \[
        2 + x_n \leq 4 \implies \sqrt{2+x_n} \leq 4 \implies x_{n+1} \leq 4
        \] Since \seq{x}{n} is monotone and bounded, the monotone convergence theorem tells us that $(x_n)\rightarrow x$. Equating the limits on both sides gives \[
        x = \sqrt{2+x} \implies x^2 - x - 2 = 0 \implies x = \frac12 \pm \frac32
        \] Since $x > 0$, we have $x=2$.

        \item Let $x_1 = \sqrt2$ and $x_{n+1} = \sqrt{2x_n}$, clearly $x_2 > x_1$. Suppose $x_{n+1} > x_n$, then \[
        2x_{n+1} > 2x_n \implies \sqrt{2x_{n+1}} > \sqrt{2x_n} \implies x_{n+2} > x_{n+1}\] showing that \seq{x}{n} is monotonically increase. Next we show that $x_n \leq 2$, which is clearly true for $x_1$. Suppose $x_n \leq 2$, then \[
        2x_n \leq 4 \implies x_{n+1} = \sqrt{2x_n} \leq 2\] Hence, $x_n$ is bounded. By monotone convergence theorem, the sequence converges, hence $(x_n) \rightarrow x$. Taking limits on both sides, \[
        x = \sqrt{2x} \implies x^2 -2x = 0 \implies x = 1 \pm 1\]Since $x >0$, then $x = 2$.
    }}
}

\bx{ % 2.4.4
    \ea{
        \item In Section 1.4, we used the Axiom of Completeness (AoC) to prove the Archimedean Property of $\mathbb{R}$ (Theorem 1.4.2). Show that the Monotone Convergence Theorem can also be used to prove the Archimedean Property without making any use of AoC.
        \item Use the Monotone Convergence Theorem to supply a proof for the Nested Interval Property (Theorem 1.4.1) that doesn't make use of AoC
    }
    \noindent These two results suggest that we could have used the Monotone Convergence Theorem in place of AoC as out starting axiom for building a proper theory of the real numbers.
    \sol{\ea{
        \item MCT tells us $(1/n)$ converges, obviously it must converge to zero, giving us $|1/n - 0| = 1/n < \epsilon$ for any $\epsilon>0$, which is the Archimedean Property.
        \item We have $I_n = [a_n, b_n]$ with $a_n \leq b_n$ since $I_n \neq \emptyset$. Since $I_{n+1} \subseteq I_n$, we must have $b_{n+1} \leq b_n$ and $a_{n+1} \geq a_n$, the MCT tells us that $(a_n)\rightarrow a$ and $(b_n) \rightarrow b$. By Order Limit Theorem, we have $a \leq b$, hence $a \in I_n$ for all $n \in \mathbb{N}$, meaning $a \in \bigcap^\infty_{n=1} I_n \neq \emptyset$.
    }}
}

\bx{ % 2.4.5
    \textbf{(Calculating Square Roots)} Let $x_1 = 2$, and define \[
    x_{n+1} = \frac12 \left(x_n + \frac2{x_n}\right).\] \ea{
        \item Show that $x_n^2$ is always greater than or equal to 2, and then use this to prove thath $x_n - x_{n+1} \geq 0$. Conclude that $\lim x_n = \sqrt2$.
        \item Modify the sequence \seq{x}{n} so that it converges to $\sqrt{c}$.
    }

    \sol{\ea{
        \item Clearing $x_1 \geq 2$. For any $n$, \begin{align*}
            x_{n+1}^2 & = \left[\frac12 \left(x_n + \frac{2}{x_n}\right)\right]^2 \\
            & = \frac14 \left(x_n^2 + 4 + \frac4{x_n^2}\right) \\
            & = \frac14 \left(x_n^2 - 4 + \frac{4}{x_n^2}\right) + 2 \\
            & = \frac14 \left(x_n + \frac{2}{x_n}\right) + 2 \geq 2
        \end{align*}
        Now we show $x_n - x_{n+1} \geq 0$. \begin{align*}
            x_n - x_{n+1} & = x_n - \frac12 \left(x_n + \frac2{x_n}\right) \\
            & = \frac12 x_n - \frac1{x_n} \geq 0 \because x_n \geq 0
        \end{align*}
        MCT tells us that $(x_n) \rightarrow x$. By taking limits on both sides, \[
        x = \frac12 \left(x + \frac2x\right) \implies \frac12 x^2 = 1 \implies x =\pm \sqrt2\]
        Since $x_n \geq 0$, $x =\sqrt2$.

        \item \[x_{n+1} = \frac12 \left(x_n + \frac{c}{x_n}\right)\]
        Proof of this is similar to (a) and left as an exercise for the reader.
    }}
}

\bx{ % 2.4.6
    \textbf{(Arithmetic-Geometric Mean)} \ea{
        \item Explain why $\sqrt{xy} \leq (x+y) / 2$ for any two positive real numbers $x$ and $y$. (The geometric mean is always less than the arithmetic mean.)
        \item Now let $0 \leq x_1 \leq y_1$ and define \[
        x_{n+1} = \sqrt{x_ny_n} \quad \text{and} \quad y_{n+1} = \frac{x_n + y_n}{2}\] Show $\lim x_n$ and $\lim y_n$ both exist and are equal.
    }
    \sol{\ea{
        \item We have \[
        \sqrt{xy} \leq \frac{x+y}{2} \iff 4xy \leq x^2 + 2xy + y^2 \iff 0 \leq (x-y)^2\]
        \item As shown above, if $\lim x_n = x$ and $\lim y_n = y$, then $x_n = y_n$ is the only fixed point, hence $x = y$. Thus we only need to show both sequences converge.
        
        The inequality $0 \leq x_n \leq y_n$ is always true. It is obvious both terms will always be positive. Suppose $x_n \leq y_n$, then by (a), $x_{n+1} \leq y_{n+1}$ true.

        To continue, $x_n \leq y_n$ implies $(x_n +y_n)/2 = y_{n+1} \leq y_n$. Similarly, $\sqrt{x_ny_n} = x_{n+1} \geq x_n$. This means both sequences are monotone and bounded by each other, hence Monotone Convergence Theorem tells us both sequences converge.
    }}
}

\bx{ % 2.4.7
    \textbf{(Limit Superior)} Let \seq{a}{n} be a bounded sequence. \ea{
        \item Prove that the sequence defined by $y_n = \sup \{a_k : k \geq n\}$ converges.
        \item The \textit{limit superior} of \seq{a}{n}, or $\lim \sup a_n$, is defined by \[
        \lim \sup a_n = \lim y_n,
        \] where $y_n$ is the sequence from part (a) of this exercise. Provide a reasonable definition for $\lim \inf a_n$ and briefly explain why it always exists for any bounded sequence.
        \item Prove that $\lim \inf a_n \leq \lim \sup a_n$ for every bounded sequence, and give an example of a sequence for which the inequality is strict.
        \item Show that $\lim\inf a_n = \lim \sup a_n$ \itiff $\lim a_n$ exists. In this case, all three share the same value.
    }
    \sol{ \ea{
        \item \seq{y}{n} is decreasing, hence MCT tells us it converges.
        \item The \textit{limit inferior} of \seq{a}{n}, or $\lim \inf a_n$, is defined by $\lim\inf a_n = \lim x_n$, where $x_n = \inf\{a_k: k \geq n\}$.
        
        Since \seq{x}{n} is increasing, MCT tells us it converges, hence the limit inferior exists.
        \item Obviously, $\inf\{a_k : k \geq n\} \leq \sup\{a_k : k \geq n\}$, so Order Limit Theorem implies $\lim\inf a_n \leq \lim\sup a_n$. 
        
        The equality is strict when the series does not converge, for example $a_n = (-1)^n$.\
        \item $(\Rightarrow)$ If $\lim \inf a_n = \lim \sup a_n$, then Squeeze Theorem implies that $a_n$ converges to the same value, since $\inf\{a_{k\geq n}\} \leq a_n \leq \sup\{a_{k\geq n}\}$.
        
        $(\Leftarrow)$ Assume that $\lim a_n = a$. For $\epsilon > 0$, we can find $N > 0$ such that for all $k > N$, $|a_k - a| < \epsilon$. This means \[
        a - \epsilon < x_N \leq x_k \leq y_k \leq y_N < a + \epsilon\]
        implying that both $x_n$ and $y_n$ converge to $a$.
        }}
}