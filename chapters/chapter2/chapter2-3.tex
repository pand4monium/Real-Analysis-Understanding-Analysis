\section{The Algebraic and Order Limit Theorem}

\bx{ % Exercise 2.3.1
    Let $x_n \geq 0$ for all $n \in \mathbb{N}$.\ea{
        \item If $(x_n) \rightarrow 0$, show that $(\sqrt{x_n}) \rightarrow 0$.
        \item If $(x_n) \rightarrow x$, show that $(\sqrt{x_n}) \rightarrow \sqrt{x}$.
    }
    \sol{\ea{
        \item For $\epsilon^2 > 0$, there exists $N > 0$, such that for all $n > N$, $x_n < \epsilon^2$, implying $\sqrt{x_n} < \epsilon$.
        \item We want to show that $|\sqrt{x_n} - \sqrt{x}| < \epsilon$. Multiplying by $(\sqrt{x_n}+\sqrt{x})$ gives us $|x_n - x| < (\sqrt{x_n}+\sqrt{x}) \epsilon$. Since $(x_n)$ converges, it is bounded, giving $|x_n| \leq M$, then $\sqrt{x_n} \leq \sqrt{M}$. \[
        |x_n - x| < (\sqrt{x_n} + \sqrt{x}) \epsilon \leq (M + \sqrt{x}) \epsilon
        \] Setting $|x_n - x| < (M + \sqrt{x}) \epsilon$ for $n > N$, we get \[
        |\sqrt{x_n} - \sqrt{x}| = \frac{|x_n - x|}{\sqrt{x_n} + \sqrt{x}} \leq \frac{|x_n - x|}{M + \sqrt{x}} = \epsilon
        \]thus completing the proof.
    }}
}

\bx{ % Exercise 2.3.2
    Using only Definition 2.2.3, prove that if $(x_n) \rightarrow 2$, then \ea{
        \item $(\frac{2x_n - 1}{3}) \rightarrow 1$;
        \item $(1/x_n) \rightarrow 1/2$;
    } (For this exercise the Algebraic Limit Theorem is off-limits, so to speak.)
    \sol{\ea{
        \item We have $|\frac{2x_n - 1}{3} - 1| = |\frac{2}{3}x_n - \frac43| = \frac23 |x_n - 2|$. Setting $|x_n - 2| < \frac32\epsilon$ for all $n > N$, we get $|\frac{2x_n-1}{3} - 1| < \epsilon$.
        \item We set $N$ such that $|x_n - 2| < $. Since $x_n$ is at least 1, we can bound $|1/x_n| \leq 1$, giving \[
        |1/x_n - 1/2| = \frac{|2 - x_n|}{|2x_n|} \leq \frac{|x_n - 2|}{2} \leq \frac\epsilon2 < \epsilon.
        \]
    }}
}

\bx{ % Exercise 2.3.3
    \textbf{(Squeeze Theorem)} Show that if $x_n \leq y_n \leq z_n$ for all $n \in \mathbb{N}$, and if $\lim x_n = \lim z_n = l$, then $\lim y_n = l$ as well.
    \sol{
        Let $\epsilon > 0$, set $N>0$ so that $|x_n - l| < \epsilon/3$ and $|z_n - l| < \epsilon/3$. By triangle inequality, we get \[
        |z_n - x_n| \leq |z_n - l| + |l - x_n| < 2\epsilon/3
        \]Since $z_n \geq y_n \geq x_n$, $|y_n - x_n| = y_n - x_n \leq z_n - x_n$. So by triangle inequality, \[
        |y_n - l| \leq |y_n - x_n| + |x_n - l| \leq |z_n - x_n| + |x_n - l| < 2\epsilon/3 + \epsilon/3 = \epsilon\]
    }
}

\bx{ % Exercise 2.3.4
    Let $(a_n) \rightarrow 0$, and use the Algebraic Limit Theorem to compute each of the following limits (assuming the fractions are always defined): \ea{
        \item $\lim \left( \frac{1+2a_n}{1+3a_n-4a_n^2} \right)$
        \item $\lim \left( \frac{(a_n + 2)^2 - 4}{a_n}\right)$
        \item $\lim \left( \frac{\frac{2}{a_n}+3}{\frac{1}{a_n}+5}\right)$.
    }
    \sol{\ea{
        \item By ALT, \begin{align*}
            \lim \left( \frac{1+2a_n}{1+3a_n-4a_n^2} \right) &= \frac{\lim (1 + 2a_n)}{\lim (1 + 3 a_n - 4 a_n^2)} \\
            & = \frac{1 + 2 \lim a_n}{1+3\lim a_n - 4(\lim a_n)^2} \\ 
            & = 1
        \end{align*}
        \item By ALT, \begin{align*}
            \lim \left( \frac{(a_n + 2)^2 - 4}{a_n}\right) & = \lim \left( \frac{a_n^2 + 4a_n}{a_n}\right) \\
            & = \lim (a_n + 4) = 4 + \lim a_n = 4
        \end{align*}
        \item By ALT, \begin{align*}
            \lim \left( \frac{\frac{2}{a_n}+3}{\frac{1}{a_n}+5}\right) & = \lim \left( \frac{2 + 3a_n}{1 + 5a_n}\right) \\
            & = \frac{\lim(2 + 3a_n)}{\lim(1 + 5a_n)} \\
            & = \frac{2 + 3\lim a_n}{1 + 5 \lim a_n} = 2
        \end{align*}
    }}
}

\bx{ % Exercise 2.3.5
    Let $(x_n)$ and $(y_n)$ be given, and define $(z_n)$ to be the "shuffled" sequence $(x_1, y_1, x_2, y_2, x_3, y_3, \dots, x_n, y_n, \dots)$. Prove that $(z_n)$ is convergent if and only if $(x_n)$ and $(y_n)$ are both convergent with $\lim x_n = \lim y_n$.
    \sol{
        Obviously, if $\lim x_n = \lim y_n = l$, then for $\epsilon>0$, we can set $N>0$ such that $|x_n - l|< \epsilon$ and $|y_n - l|< \epsilon$ for $n > N$, thus for all $n > 2N$, $|z_n - l|< \epsilon$ will be true.

        Conversely, if $(z_n) \rightarrow l$, for $\epsilon > 0$, we can set $2N > 0$ such that for all $n > 2N$, $|z_n - l| < \epsilon$, implying that for $n > N$, $|x_n - l| < \epsilon$ and $|y_n - l| < \epsilon$ will also be true.
    }
}

\bx{ % Exercise 2.3.6
    Consider the sequence given by $b_n = n - \sqrt{n^2 + 2n}$. Taking $(1/n) \rightarrow 0$ as given, and using both the Algebraic Limit Theorem and the result in Exercise 2.3.1, show $\lim b_n$ exists and find the value of the limit.
    \sol{
        Let $(b_n) \rightarrow b$. To remove the radical from the term, consider: \[
        (n - \sqrt{n^2+2n})(n + \sqrt{n^2+2n}) = n^2 -(n^2+2n) = -2n 
        \] By ALT, \begin{align*}
        b = \lim \left( \frac{-2n}{n + \sqrt{n^2 + 2n}}\right)
        = \lim \frac{-2}{1 + \sqrt{1 +2/n}}
        = \frac{-2}{1 +\sqrt{1 + 2 \lim(1/n)}}
        = \frac{-2}{1+ \sqrt{1+ 0}}
        = -1
        \end{align*}thus showing that the limit exists and $(b_n) \rightarrow 1$.
    }
}

\bx{ % Exercise 2.3.7
    Give an example of each of the following, or state that such a request is impossible by referencing the proper theorem(s): \ea{
        \item sequences $(x_n)$ and $(y_n)$, which both diverge but whose sum $(x_n + y_n)$ converges;
        \item sequences $(x_n)$ and $(y_n)$, where $(x_n)$ converges, $(y_n)$ diverges, and $(x_n + y_n)$ converges;
        \item a convergent sequence $(b_n)$ with $b_n \neq 0$ for all $n$ such that $(1/b_n)$ diverges;
        \item an unbounded sequence $(a_n)$ and a convergent sequence $(b_n)$ with $(a_n-b_n)$ bounded;
        \item two sequences $(a_n)$ and $(b_n)$, where $(a_nb_n)$ and $(a_n)$ converge but $(b_n)$ does not.
    }
    \sol{\ea{
        \item Consider $x_n = n$ and $y_n = -n$ but $x_n + y_n = 0$ for all $n \in \mathbb{N}$.
        \item Impossible. By ALT, $\lim y_n = \lim (x_n + y_n) - \lim x_n$ must converge, which is a contradiction.
        \item Consider $b_n = 1/n$, then $1/b_n = n$, so $(1/b_n) \rightarrow \infty$.
        \item Impossible. Let $|b_n| \leq M$ and $|a_n - b_n| \leq N$. By triangle inequality, \[
            |a_n| \leq |a_n - b_n| + | b_n| \leq M+N
        \], hence is also bounded, which is a contradiction.
        \item  Consider $a_n = 1/n$ and $b_n = n$, then $a_nb_n = 1$.
    }}
}