\section{The Cauchy Criterion}

\bx{ % 2.6.1
    Supply a proof for Theorem 2.6.2.
    \sol{
        Suppose $(x_n) \rightarrow x$, hence for $\epsilon > 0$, $|x - x_n| < \epsilon/2$ when $n > N$. Similarly, $|x_m - x| < \epsilon/2$ for $m > N$. By Triangle Inequality, \[
        |x_n - x_m| \leq |x_n - x| + |x - x_m| < \epsilon/2 + \epsilon/2 = \epsilon\]
    }
}

\bx{ % 2.6.2
    Give an example of each of the following, or argue that such a request is impossible. \ea{
        \item A Cauchy sequence that is not monotone.
        \item A Cauchy sequence with an unbounded subsequence.
        \item A divergent monotone sequence with a Cauchy subsequence.
        \item An unbounded sequence containing a subsequence that is Cauchy.
    }
    \sol{\ea{
        \item $x_n = (-1)^n/n$ converges and hence Cauchy but not monotone.
        \item Impossible. Cauchy sequences must converge and hence bounded.
        \item Impossible. A Cauchy subsequence will converge, implying that it is bounded. Since the sequence is montone, it is also bounded, thus MCT tells us it converges.
        \item $(2, 1/2, 3, 1/3, \dots)$ has the subsequence $(1/2, 1/3, \dots)$ which is Cauchy.
    }}
}