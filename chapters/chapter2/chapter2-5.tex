\section{Subsequences and the Bolzano-Weierstrass Theorem}

\bx{ % 2.5.1
    Give an example of each of the following, or argue that such a request is impossible. \ea{
        \item A sequence that has a subsequence that is bounded but contains no subsequence that converges.
        \item A sequence that does not contain 0 or 1 as a term but contains subsequences converging to each of these values.
        \item A sequence that contains subsequences converging to every point in the infinite set $\{1, 1/2, 1/3, 1/4, 1/5, \dots\}$.
        \item A sequence that contains subsequences converging to every point in the infinite set $\{1, 1/2, 1/3, 1/4, 1/5, \dots\}$, and no subsequences converging to points outside of this set.
    }

    \sol{\ea{
        \item Impossible, the Bolzano-Weierstrass theorem tells us a convergent subsequence of the subsequence exists, and that sub-sub-sequence is also a subsequence of the original sequence.
        \item $(1 +1/n) \rightarrow 1$ and $(1/n) \rightarrow 0$. By interweaving the terms, we get $(1/2, 3/2, 1/3, 4/3, \dots)$ that contains subsequences that converge to both 0 and 1.
        \item Consider the sequence where there is an infinite number of terms for each element in the set \[
        (1, 1/2, 1, 1/2, 1/3, 1, 1/2, 1/3, 1/4, \dots)\]
        \item Impossible, the sequence must converge to zero which is not in the set.
        
        Let $\epsilon > 0$ be arbituary, pick $N>0$ large enough such that $1/n < \epsilon /2$ for $n > N$. We can find a subsequence $(b_m) \rightarrow 1/n$ meaning $|b_m - 1/n| < \epsilon/2$ for some $m$. By triangle inequality, we get \[
        |b_m - 0| \leq |b_n - 1/n| + |1/n - 0| < \epsilon/2 + \epsilon/2 = \epsilon
        \]therefore we have found a number $b_m$ in the sequence $a_m$ with $|b_m| < \epsilon$. Repeating this process for any $\epsilon$ allows us to construct a subsequence that converges to zero.
    }}
}

\bx{ % 2.5.2
    Decide whether the following propositions are true or false, providing a short justification for each conclusion. \ea{
        \item If every proper subsequence of \seq{x}{n} converges, then \seq{x}{n} converges as well.
        \item If \seq{x}{n} contains a divergent subsequence, then \seq{x}{n} diverges.
        \item If \seq{x}{n} is bounded and diverges, then there exist two subsequences of \seq{x}{n} that converge to different limits.
        \item If \seq{x}{n} is monotone and contains a convergent subsequence, then \seq{x}{n} converges.
    }
    \sol{\ea{
        \item True, removing the first term gives us the proper subsequence $(x_2, x_3, \dots)$ which converges. This implies that \seq{x}{n} converges to the same value, since changing the index of the terms does not change the limit behaviour of the sequence.
        \item True, the contrapositive of the statement is "If $(x_n)$ converges, then every subsequence converges to the same value", which is true.
        \item True. By BWT, the bounded sequence $(x_n)$ has a convergent subsequence as it is bounded. Let $(a_n)$ be one such subsequence with limit $a$. Since $(x_n)$ does not converge to $a$, there exists $\epsilon > 0$ such that there are infinitely many terms of $(x_n)$ that lie outside the interval $(a-\epsilon, a + \epsilon)$. Let \seq{b}{n} be the sequence of those terms, and since it is bounded, BWT tells us we can find another convergent subsequence \seq{c}{n} with limit of c. Since the terms of \seq{c}{n} lie outside the $\epsilon$-neighbourhood of $a$, $c \neq a$. Hence we have found two subsequences that converge to different limits.
        \item True. The subsequence \seq{x}{n_k} converges, means that it is bounded $|x_{n_k}| \leq M$. Suppose $(x_n)$ is increasing, then $x_n$ is bounded since we can pick a $k$ so that $n_k > n$, we have $x_n \leq x_{n_k} \leq M$, thus converging. A similar argument can done if \seq{x}{n} is decreasing by taking the negative of each terms as a new sequence. 
    }}
}