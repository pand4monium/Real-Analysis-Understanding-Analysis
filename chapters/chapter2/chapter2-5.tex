\section{Subsequences and the Bolzano-Weierstrass Theorem}

\bx{ % 2.5.1
    Give an example of each of the following, or argue that such a request is impossible. \ea{
        \item A sequence that has a subsequence that is bounded but contains no subsequence that converges.
        \item A sequence that does not contain 0 or 1 as a term but contains subsequences converging to each of these values.
        \item A sequence that contains subsequences converging to every point in the infinite set $\{1, 1/2, 1/3, 1/4, 1/5, \dots\}$.
        \item A sequence that contains subsequences converging to every point in the infinite set $\{1, 1/2, 1/3, 1/4, 1/5, \dots\}$, and no subsequences converging to points outside of this set.
    }

    \sol{\ea{
        \item Impossible, the Bolzano-Weierstrass theorem tells us a convergent subsequence of the subsequence exists, and that sub-sub-sequence is also a subsequence of the original sequence.
        \item $(1 +1/n) \rightarrow 1$ and $(1/n) \rightarrow 0$. By interweaving the terms, we get $(1/2, 3/2, 1/3, 4/3, \dots)$ that contains subsequences that converge to both 0 and 1.
        \item Consider the sequence where there is an infinite number of terms for each element in the set \[
        (1, 1/2, 1, 1/2, 1/3, 1, 1/2, 1/3, 1/4, \dots)\]
        \item Impossible, the sequence must converge to zero which is not in the set.
        
        Let $\epsilon > 0$ be arbituary, pick $N>0$ large enough such that $1/n < \epsilon /2$ for $n > N$. We can find a subsequence $(b_m) \rightarrow 1/n$ meaning $|b_m - 1/n| < \epsilon/2$ for some $m$. By triangle inequality, we get \[
        |b_m - 0| \leq |b_n - 1/n| + |1/n - 0| < \epsilon/2 + \epsilon/2 = \epsilon
        \]therefore we have found a number $b_m$ in the sequence $a_m$ with $|b_m| < \epsilon$. Repeating this process for any $\epsilon$ allows us to construct a subsequence that converges to zero.
    }}
}

\bx{ % 2.5.2
    Decide whether the following propositions are true or false, providing a short justification for each conclusion. \ea{
        \item If every proper subsequence of \seq{x}{n} converges, then \seq{x}{n} converges as well.
        \item If \seq{x}{n} contains a divergent subsequence, then \seq{x}{n} diverges.
        \item If \seq{x}{n} is bounded and diverges, then there exist two subsequences of \seq{x}{n} that converge to different limits.
        \item If \seq{x}{n} is monotone and contains a convergent subsequence, then \seq{x}{n} converges.
    }
    \sol{\ea{
        \item True, removing the first term gives us the proper subsequence $(x_2, x_3, \dots)$ which converges. This implies that \seq{x}{n} converges to the same value, since changing the index of the terms does not change the limit behaviour of the sequence.
        \item True, the contrapositive of the statement is "If $(x_n)$ converges, then every subsequence converges to the same value", which is true.
        \item True. By BWT, the bounded sequence $(x_n)$ has a convergent subsequence as it is bounded. Let $(a_n)$ be one such subsequence with limit $a$. Since $(x_n)$ does not converge to $a$, there exists $\epsilon > 0$ such that there are infinitely many terms of $(x_n)$ that lie outside the interval $(a-\epsilon, a + \epsilon)$. Let \seq{b}{n} be the sequence of those terms, and since it is bounded, BWT tells us we can find another convergent subsequence \seq{c}{n} with limit of c. Since the terms of \seq{c}{n} lie outside the $\epsilon$-neighbourhood of $a$, $c \neq a$. Hence we have found two subsequences that converge to different limits.
        \item True. The subsequence \seq{x}{n_k} converges, means that it is bounded $|x_{n_k}| \leq M$. Suppose $(x_n)$ is increasing, then $x_n$ is bounded since we can pick a $k$ so that $n_k > n$, we have $x_n \leq x_{n_k} \leq M$, thus converging. A similar argument can done if \seq{x}{n} is decreasing by taking the negative of each terms as a new sequence. 
    }}
}

\bx{ % 2.5.3
    \ea{
        \item Prove that if an infinite series converges, then the associative property holds. Assume $a_1+a_2+a_3+a_4+a_5 + \dots$ converges to a limit $L$ (i.e., the sequence of partial sums $(s_n) \rightarrow L$). Show that any regrouping of the terms \[
        (a_1 + a_2 + \dots + a_{n_1}) + (a_{n_1+1}+\dots+a_{n_2}) + (a_{n_2+1}+ \dots + a_{n_3})+\dots\] leads to a series that also converges to $L$.
        \item Compare this result to the example discussed at the end of Section 2.1 where the infinite addition was not to be associative. Why doesn't our proof in (a) apply to this example?
    }
    \sol{\ea{
        \item Let $s_n$ be the original partial sums and $s'_m$ be the regrouping. Since \seq{s'}{m} is a subsequence of \seq{s}{n}, $(s_n) \rightarrow L$ implies $(s'_n) \rightarrow L$.
        \item The converse is not true. $(s'_m)$ converging does not imply that $(s_n)$ converges.
    }}
}

\bx{ % 2.5.4
    The Bolzano-Weierstrass Theorem is extremely important, and so is the strategy employed in the proof. TO gain some more experience with this technique, assume the Nested Interval Property is true an duse it to provide a proof of the Axiom of Completeness. TO prevent the argument from being circular, assume also that $(1/2^n)\rightarrow 0$. (Why precisely is this last assumption needed to avoid circularity?)
    \sol{
        Let $A$ be a bounded set. We are going to conduct a binary search for $\sup A$ and then use NIP to prove the limit exists.


        Let $M$ be an upper bound on A, and pick any $L \in A$ as our starting lower bound for $\sup A$. We define $I_1 = [L, M]$. Doing a binary search, gives us $I_{n+1} \subseteq I_n$ with length proportional to $(1/2)^n$. Applying NIP gives us $\bigcap^\infty_{n=1} I_n \neq \emptyset$. As the length goes to zero, there exists a single $s \in \bigcap^\infty_{n=1}I_n$, which must be the least upper bound since, \er{
            \item $s \geq L_n$ implies $s$ is an upper bound
            \item $s \leq M_n$ implies $s$ is the least upper bound
        }


        The assumption that $(1/2^n) \rightarrow 0$ is necessary because the Archimedian Property was proved using the Axiom of Completeness.
    }
}

\bx{ % 2.5.5
    Assume \seq{a}{n} is a bounded sequence with the property that every convergent subsequence of \seq{a}{n} converges to the same limit $a \in \mathbb{R}$. Show that \seq{a}{n} must converge to $a$.
    \sol{
        Suppose \seq{a}{n} does not converge to $a$. For some $\epsilon > 0$ wee can find a subsequence \seq{b}{n} satisfying $|b_n - a| \geq \epsilon$ for all $n\in \mathbb{N}$. Since \seq{a}{n} is bounded, so must \seq{b}{n}. BWT tells us there must be a convergent subsequence of \seq{b}{n} that converges to $b$. Evidently, $b \neq a$ as $|b_n - a| \geq \epsilon$, leading to a contradiction as not every convergent subsequence of \seq{a}{n} converges to the same limit.
    }
}

\bx{ % 2.5.6
    Use a similar strategy to the one in Example 2.5.3 to show that $\lim b^{1/n}$ exists for all $b \geq 0$ and find the value of the limit. (The results in Exercise 2.3.1 may be assumed.)
    \sol{
        To show that the limit exists, we need to apply the Monotone Convergence Theorem. Consider the three cases \er{
            \item Case 1: $b > 1$, then $b^{1/n}$ is decreasing as $\frac1{n+1} < \frac1n$ implies $\frac{\ln b}{n+1} < \frac{\ln b}{n}$ hence $b^{1/(n+1)} < b^{1/n}$. Also, $b^{1/n} \iff b > 1^n$, hence bounded.
            \item Case 2: $b < 1$, then MCT applies for similar reason as Case 1.
            \item Case 3: $b=1$. This case is trivial.
        }
        Monotone Convergence Theorem applies, hence we know the limit exists, then $\lim b^{1/n} = c$. Consider the subsequence: \[
        b^\frac{1}{2n} = \sqrt{b^{1/n}}\] Using Exercise 2.3.1, we know $\lim \sqrt{b^{1/n}} = \sqrt{c}$ and since subsequences converge to the same limit:\[
        c = \sqrt{c}\] Hence $c=0$ or $1$. For $b >1$, the only possible limit is $c=1$. For $b<1$, the increasing sequence is bounded by $b^{1/n} \geq b$. Hence, for $0<b<1$, $c = 1$. For the case when $b = 0$, $c=0$ is trivial.
    }
}

\bx{ %  2.5.7
    Extend the result proved in Example 2.5.3 to the case $|b| < 1$; that is, show $\lim (b^n)$ if and only if $-1<b<1$.
    \sol{
        ($\Rightarrow$) Given $\lim (b^n) = 0$, \AFSOC $|b| \geq 1$. Then $\lim b^n \neq b$. (Diverges for $b \neq 1$)
        
        ($\Leftarrow$) If $|b| < 1$, the  $|b^n|<1$, thus $b^n$ is bounded. By Exercise 2.5.3, it is decreasing and monotone convergence theorem applies. To find the limit, equating the terms gives us $b^{n+1} = b^n$, thus $b=0$ or $b=1$. Since $b$ is strictly decreasing, we have $b = 0$.
    }
}

\bx{ % 2.5.8
    Another way to prove the Bolzano-Weierstrass Theorem is to show that every sequence contains a monotone subsequence. A useful device in this endeavor is the notion of \textit{peak term}. Given a sequence \seq{x}{n}, a particular $x_m$ is a peak term if no later term in the sequence exceeds it; i.e., if $x_m \geq x_n$ for all $n \geq m$. \ea{
        \item Find examples of sequences with zero, one and two peak terms. Find an example of a sequnce with infinitely many peak terms that is not monotone. 
        \item Show that every sequence contains a monotone subsequence and explain how this furnishes a new proof of the Bolzano-Weierstrass Theorem.
    }
    \sol{\ea{
        \item $(1, 2, 3, \dots )$, $(1, 1/2, 2/3, \dots)$ and $(2, 1, 1/2, 2/3 \dots )$ has zero, one and two peak terms respectively. The sequence $(1, -1/2, 1/3, -1/4, \dots)$ has infinitely many peak terms, but is not montone.
        \item If there is an infinite number of peak terms. Each peak term is, by definition, lower than the last. We can take the subsequence of peak terms, which is strictly increasing, to find a monotone subsequence.
        
        If there is a finite number of peak term, let the last peak term be at the index $k$. Consider the subsequence of terms after the last peak term. Since there is no peak terms after $x_k$, then $x_m \geq x_n$ for any $m > n > k$. Hence, the subsequence is monotone increasing.

        If the original sequence is bounded, and we can find a monotone subsequence, MCT tells us that this subsequence converges, proving BWT.
    }}
}

\bx{ % 2.5.9
    Let \seq{a}{n} be a bounded sequence, and define the set \[
    S = \{x \in \mathbb{R}: x < a_n \text{ for infinitely many terms } a_n\}\]
    Show that there exists a subsequence \seq{a}{n_k} converging to $s = \sup S$. (This is a direct proof of the Bolzano-Weierstrass Theorem using the Axiom of Completeness.)
    \sol{
        For every $\epsilon>0$, there exists an $x \in S$ with $x > s-\epsilon$, implying $|s-x| < \epsilon$. There for we get can close to $s = \sup S$. WE can hence pick $x_n \in S$ such that $|x_n - s| < 1/n$ for all $n \in \mathbb{N}$. Hence picking $N > 1/\epsilon$, we get $|x_n -s|< \epsilon$ for all $n > N$.
    }
}