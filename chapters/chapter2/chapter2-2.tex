\section{The Limit of a Sequence}

\bx{ % Exercise 2.2.1
    What happens if we reverse the order of the quantifiers in Definition 2.2.3?

    \textit{Definition:} A sequence ($x_n$) \textit{verconges} to $x$ if \textit{there exists} an $\epsilon > 0$ such that \textit{for all} $N \in \mathbb{N}$ it is true that $n \geq N$ implies $|x_n - x| < \epsilon$.

    Give an example of a vercongent sequence. Is there an example of a vercongent dequence that is divergent? Can a sequence verconge to two different values? What exactly is being described in this strange definition?
    \sol{
        \er{
            \item Consider the sequence $x_n = \sin x$. $(x_n)$ verconges to $0$ as we can choose $\epsilon = 1$ such that $|x_n - 0| < \epsilon$ for any value of $n \in \mathbb{N}$.
            \item There is not vercongent sequence that diverges. THe proof is obvious and left as an exercise for the reader.
            \item The sequence in (i) verconges to $x \in \mathbb{R}$ by choosing $\epsilon = x + 2$.
            \item The definition of verconges described a bounded sequence.
        }
    }
}

\bx{ % Exercise 2.2.2
    Verify, using the definition of convergence of a sequence, that the following sequences converge to the proposed limit. \ea{
        \item $\lim \frac{2n+1}{5n+4} = \frac25$.
        \item $\lim \frac{2n^2}{n^3+3} = 0$.
        \item $\lim \frac{\sin(n^2)}{\sqrt[3]n}$.
    }
    \sol{
        \ea{
            \item For $\epsilon > 0$, we can choose $N = 1/\epsilon$, such that for all $n > N$, \[
                \left| \frac{2n+1}{5n+4} - \frac25 \right| = \left| \frac{-3}{5(5n+4)} \right| = \frac{3}{5(5n+4)} < \frac1n < \epsilon \implies n > \frac1\epsilon
            \]
            \item For $\epsilon > 0$, we can choose $N = 2/\epsilon$, such that for all $n > N$, \[
                \left| \frac{2n^2}{n^3+3} - 0\right| = \frac{2n^2}{n^3+3} < \frac{2n^2}{n^3} = \frac2n < \epsilon \implies n > \frac2\epsilon
            \]
            \item For $\epsilon > 0$, we can choose $N = 1/\epsilon^3$, such that for all $n > N$, \[
                \left|\frac{\sin(n^2)}{\sqrt[3]{n}} - 0\right| = \frac{\sin(n^2)}{n^{1/3}} \leq \frac1{n^{1/3}} \implies n > \frac1{\epsilon^3}
            \]
        }
    }
}