\section{The Limit of a Sequence}

\bx{ % Exercise 2.2.1
    What happens if we reverse the order of the quantifiers in Definition 2.2.3?

    \textit{Definition:} A sequence ($x_n$) \textit{verconges} to $x$ if \textit{there exists} an $\epsilon > 0$ such that \textit{for all} $N \in \mathbb{N}$ it is true that $n \geq N$ implies $|x_n - x| < \epsilon$.

    Give an example of a vercongent sequence. Is there an example of a vercongent dequence that is divergent? Can a sequence verconge to two different values? What exactly is being described in this strange definition?
    \sol{
        \er{
            \item Consider the sequence $x_n = \sin x$. $(x_n)$ verconges to $0$ as we can choose $\epsilon = 1$ such that $|x_n - 0| < \epsilon$ for any value of $n \in \mathbb{N}$.
            \item There is not vercongent sequence that diverges. THe proof is obvious and left as an exercise for the reader.
            \item The sequence in (i) verconges to $x \in \mathbb{R}$ by choosing $\epsilon = x + 2$.
            \item The definition of verconges described a bounded sequence.
        }
    }
}

\bx{ % Exercise 2.2.2
    Verify, using the definition of convergence of a sequence, that the following sequences converge to the proposed limit. \ea{
        \item $\lim \frac{2n+1}{5n+4} = \frac25$.
        \item $\lim \frac{2n^2}{n^3+3} = 0$.
        \item $\lim \frac{\sin(n^2)}{\sqrt[3]n}$.
    }
    \sol{
        \ea{
            \item For $\epsilon > 0$, we can choose $N = 1/\epsilon$, such that for all $n > N$, \[
                \left| \frac{2n+1}{5n+4} - \frac25 \right| = \left| \frac{-3}{5(5n+4)} \right| = \frac{3}{5(5n+4)} < \frac1n < \epsilon \implies n > \frac1\epsilon
            \]
            \item For $\epsilon > 0$, we can choose $N = 2/\epsilon$, such that for all $n > N$, \[
                \left| \frac{2n^2}{n^3+3} - 0\right| = \frac{2n^2}{n^3+3} < \frac{2n^2}{n^3} = \frac2n < \epsilon \implies n > \frac2\epsilon
            \]
            \item For $\epsilon > 0$, we can choose $N = 1/\epsilon^3$, such that for all $n > N$, \[
                \left|\frac{\sin(n^2)}{\sqrt[3]{n}} - 0\right| = \frac{\sin(n^2)}{n^{1/3}} \leq \frac1{n^{1/3}} \implies n > \frac1{\epsilon^3}
            \]
        }
    }
}

\bx{ % Exercise 2.2.3
    Describe what we would have to demonstrate in order to disprove each of the following statements. \ea{
        \item At every college in the United States, there is a student who is at least seven feet tall.
        \item For all colleges in the United States, there exists a professor who gives every student a grade of either A or B.
        \item There exists a college in the United States where every student is at least six feet tall.
    }
    \sol{\ea{
        \item Find a college in the United States that has every student under seven feet.
        \item Find a college in the United States that have every professor giving at least one grade lower than a B.
        \item Find a student in each college in the United States that is under six feet
    }}
}

\bx{ %Exercise 2.2.4
    Give an example of each or state that the request is impossible. For any that are impossible, give a compelling argument for why that is the case. \ea{
        \item A sequence with an infinite number of onesthat does not converge to one.
        \item A sequence with an infinite number of oes that converges to a limit not equals to one.
        \item A divergent sequence such that for every $n \in \mathbb{N}$ it is possible to find $n$ consecutive ones somewhere in the sequence.
    }
    \sol{\ea{
        \item Consider $x_n = (-1)^n$.
        \item Impossible. Suppose $\lim x_n = a \neq 1$. Since there are infinite ones, we cannot find $N > 0$ for $\epsilon < |1-a|$.
        \item Consider the sequence $(1, 2, 1, 1, 3, 1, 1, 1, \dots)$.
    }}
}

\bx{ % exercise 2.2.5
    Let $[[x]]$ be the greatest integer less than or equal to $x$. For example, $[[\pi]] = 3$ and $ [[3]] = 3$. For each sequence, find $\lim a_n$ and verify it with the definition of convergence.\ea{
        \item $a_n = [[5/n]]$.
        \item $a_n = [[(12 + 4n) / 3n]]$.
    } Reflecting on these examples, comment on the statement following the Definition 2.2.3 that "the smaller the $\epsilon$-neighborhood, the larger $N$ may have to be."
    \sol{\ea{
        \item For all $n > 5$, we have $[[5/n]] = 0$, hence $\lim a_n = 0$.
        \item The insides converges to $4/3$ from above, so $\lim a_n = 1$.
    } Some sequences eventually reach their limit, meaning that $N$ no longer has to increase.
    }
}

\bx{ % Exercise 2.2.6
    Prove Theorem 2.2.7. To get started assume $(a_n) \rightarrow a$ and also that $(a_n) \rightarrow b$. Now argue $a = b$.
    \sol{
        \textbf{Theorem 2.2.7 (Uniqueness of Limits).} The limit of a sequence, when it exists, must be unique.

        \AFSOC that $\lim a_n = a$ and $\lim a_n = b$, such that $a \neq b$. Then for $\epsilon > 0$, choosing $n > N_a$ implies $|a_n - a| < \epsilon/2$ and choosing $n> N_b$ implies $|a_n - b| < \epsilon/2$. Let $N = max(N_a, N_b)$, hence $|a_n - a| < \epsilon/2$ and $|a_n - b| < \epsilon/2$. By triangle inequality, \[
            \epsilon = \epsilon/2 + \epsilon/2 > |a_n - a| + |a_n - b| = |a - a_n| + |a_n - b| \leq |a - b|      
        \] showing that $a = \lim a = b$, which is a contradiction.
    }
}

\bx{ % Exercise 2.2.7
    Here are two useful definitions: \er{
        \item A sequence $(a_n)$ is \textit{eventually} in a set $A \subseteq \mathbb{R}$ if there exists an $N \in \mathbb{N}$ such that $a_n \in A$ for all $n \geq N$.
        \item A sequence $(a_n)$ is \textit{Frequently} in a set $A \subseteq \mathbb{R}$ if, for every $N \in \mathbb{N}$, there exists an $n \geq N$ such taht $a_n \in A$. \ea{
            \item Is the sequence $(-1)^n$ eventually or frequently in the set $\{1\}$?
            \item Which definition is stronger? Does frequently imply eventually or does eventually imply frequently?
            \item Give an alternate rephrasing of Definition 2.2.3B using either frequently or eventually. Which is the term we want?
            \item Suppose an infinite number of terms of a sequence $(x_n)$ are equal to 2. Is $(x_n)$ necessarily eventually in the interval $(1.9, 2.1)$? Is it frequently in $(1.9, 2.1)$?
        }
    }
    \sol{\ea{
        \item $(-1)^n$ is frequently but not eventually in $\{1\}$.
        \item Eventually implies frequently, but the converse is not true.
        \item $(x_n) \rightarrow x$ \textit{if and only if} $x_n$ is eventually in any $\epsilon$-neighborhood around $x$.
        \item $(x_n)$ is frequently in $(1.9, 2.1)$ but not always eventually (considering $x_n = 2(-1)^n$).
    }}
}