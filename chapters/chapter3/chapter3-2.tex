\section{Open and Closed Sets}

\bx{ % 3.2.1
    \ea{
        \item Where in the proof of Theorem 3.2.3 part (ii) does the assumption that the collection of open sets be \textit{finite} get used?
        \item Give an example of a countable collection of open sets $\{O_1, O_2, O_3, \dots\}$ whose intersection $\bigcap_{n=1}^\infty O_n$ is closed, not empty and not all of $\mathbb{R}$.
    }
    \sol{\ea{
        \item The assumption that there a finite collection of open sets is used in the line: "Letting $\epsilon = \min \{ \epsilon_1, \epsilon_2, \dots, \epsilon_N\}$"
        \item Taking $O_n = (-1/n, 1+1/n)$ has $\bigcap_{n=1}^\infty O_n = [0, 1]$.
    }}
}

\bx{ % 3.2.2
    Let \[
        A = \left\{ (-1)^n + \frac2n: n=1, 2, 3, \dots \right\} \quad \text{and} \quad B = \{x \in \mathbb{Q}: 0 < x < 1\}.
    \]Answer the following questions for each set: \ea{
        \item What are the limit points?
        \item Is the set open? Closed?
        \item Does the set contain any isolated points?
        \item Find the closure of the set.
    }
    \sol{\ea{
        \item The set of limit points of $A$ is $\{-1, 1\}$, while that of $B$ is $[0, 1]$.
        \item Both sets are not closed as the limit point of each set is not a subset of the set. To be more specific, $\{-1, 1\} \nsubseteqq A$ and $[0,1] \nsubseteqq B$. Both sets are not open. Every interval $(a, b) \nsubseteqq A, B$
        \item Every limit point of $A$ except $1$ is isolated. $B$ has no isolated points since $B \backslash [0,1] = \emptyset$.
        \item $\overline{A} = A \cup \{-1\}$. $\overline{B} = [0,1]$.
    }}
}

\bx{ % 3.2.3
    Decide whether the following sets are open, closed, or neither. If a set is not open, find a point in the set for which there is no $\epsilon$-neighborhood contained in the set. If a set is not closed, find a limmit point that is not contained n the set. \ea{
        \item $\mathbb{Q}$.
        \item $\mathbb{N}$.
        \item $\{x \in \mathbb{R}: x \neq 0\}$.
        \item $\{1 + 1/4 + 1/9 + \dots + 1/n^2 : n \in \mathbb{N}\}$.
        \item $\{1 + 1/2 + 1/3 + \dots + 1/n : n \in \mathbb{N}\}$.
    }
    \sol{\ea{
        \item $\mathbb{Q}$ is not open and not closed. Since $\mathbb{Q}$ is dense in $\mathbb{R}$, we can find a rational number between any two rational numbers. Hence any $(a, b) \nsubseteqq \mathbb{Q}$, hence not open. We can find any irrational number to be a limit point, ie $\sqrt{2}$, hence not closed.
        \item $\mathbb{N}$ is closed as there are not limit points. $\mathbb{N}$ is not open as we can find any interval, ie $(1.5, 2.5) \nsubseteqq \mathbb{N}$.
        \item This set is not closed as $0$ is a limit point, considering the sequence of $(1/n)$ which is fully contained within the set. This set is open since every $x \in \{x \in \mathbb{R} : x \neq 0\}$ has an $\epsilon$-neighborhood around it except for zero.
        \item This set is not closed as the limit point $\pi^2 / 6$ is not contained within the set. The set is not open as it does not contain any irrationals.
        \item This set is closed as there are not limit points, but not open as it des not contain any irrationals. 
    }}
}

\bx{ % 3.2.4
    Let $A$ be nonempty and bounded above so that $s = \sup A$ exists. \ea{
        \item Show that $s \in \overline{A}$.
        \item Can an open set contain its supremum?
    }
    \sol{\ea{
        \item If $s \in A$, then $s \in \overline{A}$ and we are done. Suppose $s \notin A$, then by definition, for any $\epsilon > 0$, we can find $a \in A \cap (s - \epsilon, s)$. Setting $\epsilon = 1/n$, we can construct a sequence with a limit of $s$, thus $s \in \overline{A}$.
        \item No. Suppose an open set contains its supremum. Then for $\epsilon > 0$, the interval $V_\epsilon(s) = (s - \epsilon, s + \epsilon)$ is contained within the set $A$. This contradicts the fact that $s$ is the supremum as there exists elements $s' \in (s, s + \epsilon) \subseteq A$, where $s' > s$. Hence, an open set cannot contain its supremum.
    }}
}

\bx{ % 3.2.5
    Prove Theorem 3.2.8.
    \sol{
        To recall, Theorem 3.2.8 states that a set $F \subseteq \mathbb{R}$ is closed \textit{if and only if every Cauchy sequence contained in F has a limit that is also an element of F}.
    
        $(\Rightarrow)$ Suppose that $F \subseteq \mathbb{R}$ and that every Cauchy sequence in $F$ has a limit that is also an element in $F$. Then we have the Cauchy sequence $(x_n) \rightarrow x$ as every Cauchy sequence converges. Hence $x$ is a limit point and is contained in $F$, thus $F$ is closed.

        $(\Leftarrow)$ Let set $F \subseteq \mathbb{R}$ is closed. Suppose we have a Cauchy sequence $(x_n)$, then $(x_n) \rightarrow x$ as Cauchy sequences converge, making $x$ a limit point. Thus the limit point $x \in F$ since it is closed.
        }
}

\bx{ % 3.2.6
    Decide whether the following statements are true or false. Provide counterexamples for those that are false, and supply proofs for those that are true. \ea{
        \item An open set contains every rational number must necessarily be all of $\mathbb{R}$. 
        \item The Nested Interval Property remains true if the term "closed interval" is replaced by "closed set".
        \item Every nonempty open set contains a rational number.
        \item Every bounded infinite closed set contains a rational number.
        \item The Cantor set is closed.
    }
    \sol{\ea{
        \item False. Let $A = \mathbb{R} \backslash \{\sqrt{2}\}$, which contains every rational number and is open.
        \item False. Let $C_n = [n, \infty)$, then $\bigcap_{n=1}^\infty C_n = \emptyset$.
        \item True. Let $O$ be a nonempty open set and $x \in O$. By definition, $(x - \epsilon, x+\epsilon) \subseteq O$ and must contain a rational number since $\mathbb{Q}$ is dense in $\mathbb{R}$.
        \item False. Let $B = \{ \sqrt2 + 1/n : n \in \mathbb{N} \} \cap \{\sqrt{2}\}$, which is infinite and closed while containing only irrational numbers.
        \item True, as it is the intersection of countably many closed intervals.
    }}
}

\bx{ % 3.2.7
    Given $A \subseteq \mathbb{R}$, let $L$ be a set of all limit points of $A$. \ea{
        \item Show that the set $L$ is closed.
        \item Argue that if $x$ is a limit point of $A \cup L$, then $x$ is a limit point of $A$. Use this observation to furnish a proof for Theorem 3.2.12.
    }
    \sol{\ea{
        \item Let $(x_n) \rightarrow x$ be a sequence contained within the set $L$. Since each $x_n$ is a limit point, there must exists a sequence $(x_{n,m})$ such that $\lim_{m \rightarrow \infty} x_{n,m} = x_n$ with $x_{n,m} \in A$. We will now show that $x \in L$. 
        
        For some $\epsilon > 0$, we can find $N>0$ such that when $n > N$, $|x - x_n| < \epsilon /2$ and when $m > N$, $|x_{n,m} - x_n| < \epsilon / 2$. \[
        |x_{n,m} - x| \leq |x_{n,m} - x_n| + |x_n - x| < \epsilon/2 + \epsilon/2 = \epsilon\] and thus $x$ is also a limit point o $A$ so $x \in L$.

        \item Suppose $x$ is a limit point of $A \cup L$, then we can find a sequence $(x_n) \rightarrow x$ that is contained in $A \cup L$. We can find a subsequence of $(x_n)$ that is entirely contained within $A$ or $L$. If $(x_{n_k})$ is contained within $L$, then by part (a), $x \in L$ and hence a limit point of $A$. Otherwise, if $(x_n)$ is entirely contained within $A$, then $x$ is a limit point.
    }}
}

\bx{ % 3.2.8
    Assume $A$ is an open set and $B$ is a closed set. Determine if the following sets are definitely open, definitely closed, both, or neither. \ea{
        \item $\overline{A \cup B}$
        \item $A \backslash B = \{x \in A: x \notin B\}$
        \item $(A^c \cup B)^c$
        \item $(A \cap B) \cup (A^c \cap B)$
        \item $\overline{A}^c \cap \overline{A^c}$
    }
    \sol{\ea{
        \item Definitely closed as the closure of a set is always closed.
        \item $A \backslash B = A \cap B^c$ is definitely open as $B^c$ is open and the intersection of 2 open sets is always open.
        \item Definitely open as both $A$ and $B^c$ is closed, so the intersection is closed and the complement of the intersection will be open.
        \item Definitely closed as $(A \cap B) \cup (A^c \cap B) = B$.
        \item Open. $A^c$ is closed, so $\overline{A^c} = A^c$. Also, $\overline{A} \supseteq A$, so $\overline{A}^c \subseteq A^c$. Hence $\overline{A}^c \cap \overline{A^c} = \overline{A}^c$ is open.
    }}
}

\bx{ % 3.2.9
    \textbf{(De Morgan's Laws)} A proof of De Morgan's Laws in the case of two sets is outlined in Exercise 1.2.5. The general argument is similar. \ea{
        \item Given a collection of sets $\{E_\lambda: \lambda \in \Lambda\}$, show that \[
        \left(\bigcup_{\lambda \in \Lambda} E_\lambda\right) ^ c = \bigcap_{\lambda \in \Lambda} E_\lambda^c    \quad \text{and} \quad \left(\bigcap_{\lambda \in \Lambda} E_\lambda\right)^c = \bigcup_{\lambda \in \Lambda} E_\lambda^c\]
        \item Now, provide the details for the proof of Theorem 3.2.14.
    }
    \sol{\ea{
        \item If $x \in (\bigcup_{\lambda \in \Lambda} E_\lambda)^c$, then $x \notin \bigcup_{\lambda \in \Lambda} E_\lambda$, so $x \notin E_\lambda$ for all $\lambda \in \Lambda$. Hence, $x \in E_\lambda^c$ for all $\lambda \in \Lambda$, then $x \in \bigcap_{\lambda \in \Lambda} E_\lambda^c$, giving \[
        \left(\bigcup_{\lambda \in \Lambda} E_\lambda\right)^c \subseteq \bigcap_{\lambda \in \Lambda} E_\lambda^c\]
        For the other direction, if $x \in \bigcap_{\lambda \in \Lambda} E_\lambda^c$, then $x \in E_\lambda^c$ for all $\lambda \in \Lambda$, so $x \notin E_\lambda$ for all $\lambda \in \Lambda$. Hence $x \notin \bigcup_{\lambda \in \Lambda} E_\lambda$, then $x \in (\bigcup_{\lambda \in \Lambda} E_\lambda)^c$, giving \[
        \left(\bigcup_{\lambda \in \Lambda} E_\lambda\right)^c \supseteq \bigcap_{\lambda \in \Lambda} E_\lambda^c\]

        By showing inclusion both ways, we have shown that both sets are equal. The other relation is left as an exercise to the reader as the argument is very similar.

        \item To recall, Theorem 3.2.14 states the following: \er{
            \item The union of a finite collection of closed sets is closed.
            \item The intersection of an arbitrary collection of closed sets is closed.
        }
        Let $E_\lambda$ be closed for all $\lambda \in\Lambda$ so $E_\lambda^c$ is open. From Theorem 3.2.3, $\bigcap_{\lambda\in\Lambda} E_\lambda^c$ and $\bigcup_{\lambda\in\Lambda} E_\lambda^c$ are open, hence showing that the intersection and union of finite collection of closed sets is closed.
    }}
}

\bx{ %3.2.10
    Only one of the following three descriptions can be realized. Provide an example that illustrates the viable description, and explain why the other two cannot exist. \er{
        \item A countable \textbf{infinite} set contained in $[0,1]$ with no limit points.
        \item A countable set contained in $[0,1]$ with no isolated points.
        \item A set with an uncountable number of isolated points.
    }
    \sol{\er{
        \item Not possible as BWT states that there exists a convergent subsequence as long as the sequence is bounded.
        \item Possible. Let $A = [0,1] \cap \mathbb{Q}$.
        \item Not possible. Let $X = \{x_1, x_2, x_3, \dots\}$ be the set of all isolated points. Hence for each isolated point, there exists $V_\epsilon(x) = (x - \epsilon, x + \epsilon)$ where $V_\epsilon(x) \cap A = \{x\}$. However, in Exercise 1.5.6, we have proved that there cannot exists an uncountable collection of disjoint open intervals, meaning we cannot have an uncountable set of isolated points.
    }}
}