\section{Open and Closed Sets}

\bx{ % 3.2.1
    \ea{
        \item Where in the proof of Theorem 3.2.3 part (ii) does the assumption that the collection of open sets be \textit{finite} get used?
        \item Give an example of a countable collection of open sets $\{O_1, O_2, O_3, \dots\}$ whose intersection $\bigcap_{n=1}^\infty O_n$ is closed, not empty and not all of $\mathbb{R}$.
    }
    \sol{\ea{
        \item The assumption that there a finite collection of open sets is used in the line: "Letting $\epsilon = \min \{ \epsilon_1, \epsilon_2, \dots, \epsilon_N\}$"
        \item Taking $O_n = (-1/n, 1+1/n)$ has $\bigcap_{n=1}^\infty O_n = [0, 1]$.
    }}
}

\bx{ % 3.2.2
    Let \[
        A = \left\{ (-1)^n + \frac2n: n=1, 2, 3, \dots \right\} \quad \text{and} \quad B = \{x \in \mathbb{Q}: 0 < x < 1\}.
    \]Answer the following questions for each set: \ea{
        \item What are the limit points?
        \item Is the set open? Closed?
        \item Does the set contain any isolated points?
        \item Find the closure of the set.
    }
    \sol{\ea{
        \item The set of limit points of $A$ is $\{-1, 1\}$, while that of $B$ is $[0, 1]$.
        \item Both sets are not closed as the limit point of each set is not a subset of the set. To be more specific, $\{-1, 1\} \nsubseteqq A$ and $[0,1] \nsubseteqq B$. Both sets are not open. Every interval $(a, b) \nsubseteqq A, B$
        \item Every limit point of $A$ except $1$ is isolated. $B$ has no isolated points since $B \backslash [0,1] = \emptyset$.
        \item $\overline{A} = A \cup \{-1\}$. $\overline{B} = [0,1]$.
    }}
}