\section{Open and Closed Sets}

\bx{ % 3.2.1
    \ea{
        \item Where in the proof of Theorem 3.2.3 part (ii) does the assumption that the collection of open sets be \textit{finite} get used?
        \item Give an example of a countable collection of open sets $\{O_1, O_2, O_3, \dots\}$ whose intersection $\bigcap_{n=1}^\infty O_n$ is closed, not empty and not all of $\mathbb{R}$.
    }
    \sol{\ea{
        \item The assumption that there a finite collection of open sets is used in the line: "Letting $\epsilon = \min \{ \epsilon_1, \epsilon_2, \dots, \epsilon_N\}$"
        \item Taking $O_n = (-1/n, 1+1/n)$ has $\bigcap_{n=1}^\infty O_n = [0, 1]$.
    }}
}

