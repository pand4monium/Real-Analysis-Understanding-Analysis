\section{Open and Closed Sets}

\bx{ % 3.2.1
    \ea{
        \item Where in the proof of Theorem 3.2.3 part (ii) does the assumption that the collection of open sets be \textit{finite} get used?
        \item Give an example of a countable collection of open sets $\{O_1, O_2, O_3, \dots\}$ whose intersection $\bigcap_{n=1}^\infty O_n$ is closed, not empty and not all of $\mathbb{R}$.
    }
    \sol{\ea{
        \item The assumption that there a finite collection of open sets is used in the line: "Letting $\epsilon = \min \{ \epsilon_1, \epsilon_2, \dots, \epsilon_N\}$"
        \item Taking $O_n = (-1/n, 1+1/n)$ has $\bigcap_{n=1}^\infty O_n = [0, 1]$.
    }}
}

\bx{ % 3.2.2
    Let \[
        A = \left\{ (-1)^n + \frac2n: n=1, 2, 3, \dots \right\} \quad \text{and} \quad B = \{x \in \mathbb{Q}: 0 < x < 1\}.
    \]Answer the following questions for each set: \ea{
        \item What are the limit points?
        \item Is the set open? Closed?
        \item Does the set contain any isolated points?
        \item Find the closure of the set.
    }
    \sol{\ea{
        \item The set of limit points of $A$ is $\{-1, 1\}$, while that of $B$ is $[0, 1]$.
        \item Both sets are not closed as the limit point of each set is not a subset of the set. To be more specific, $\{-1, 1\} \nsubseteqq A$ and $[0,1] \nsubseteqq B$. Both sets are not open. Every interval $(a, b) \nsubseteqq A, B$
        \item Every limit point of $A$ except $1$ is isolated. $B$ has no isolated points since $B \backslash [0,1] = \emptyset$.
        \item $\overline{A} = A \cup \{-1\}$. $\overline{B} = [0,1]$.
    }}
}

\bx{ % 3.2.3
    Decide whether the following sets are open, closed, or neither. If a set is not open, find a point in the set for which there is no $\epsilon$-neighborhood contained in the set. If a set is not closed, find a limmit point that is not contained n the set. \ea{
        \item $\mathbb{Q}$.
        \item $\mathbb{N}$.
        \item $\{x \in \mathbb{R}: x \neq 0\}$.
        \item $\{1 + 1/4 + 1/9 + \dots + 1/n^2 : n \in \mathbb{N}\}$.
        \item $\{1 + 1/2 + 1/3 + \dots + 1/n : n \in \mathbb{N}\}$.
    }
    \sol{\ea{
        \item $\mathbb{Q}$ is not open and not closed. Since $\mathbb{Q}$ is dense in $\mathbb{R}$, we can find a rational number between any two rational numbers. Hence any $(a, b) \nsubseteqq \mathbb{Q}$, hence not open. We can find any irrational number to be a limit point, ie $\sqrt{2}$, hence not closed.
        \item $\mathbb{N}$ is closed as there are not limit points. $\mathbb{N}$ is not open as we can find any interval, ie $(1.5, 2.5) \nsubseteqq \mathbb{N}$.
        \item This set is not closed as $0$ is a limit point, considering the sequence of $(1/n)$ which is fully contained within the set. This set is open since every $x \in \{x \in \mathbb{R} : x \neq 0\}$ has an $\epsilon$-neighborhood around it except for zero.
        \item This set is not closed as the limit point $\pi^2 / 6$ is not contained within the set. The set is not open as it does not contain any irrationals.
        \item This set is closed as there are not limit points, but not open as it des not contain any irrationals. 
    }}
}

\bx{ % 3.2.4
    Let $A$ be nonempty and bounded above so that $s = \sup A$ exists. \ea{
        \item Show that $s \in \overline{A}$.
        \item Can an open set contain its supremum?
    }
    \sol{\ea{
        \item If $s \in A$, then $s \in \overline{A}$ and we are done. Suppose $s \notin A$, then by definition, for any $\epsilon > 0$, we can find $a \in A \cap (s - \epsilon, s)$. Setting $\epsilon = 1/n$, we can construct a sequence with a limit of $s$, thus $s \in \overline{A}$.
        \item No. Suppose an open set contains its supremum. Then for $\epsilon > 0$, the interval $V_\epsilon(s) = (s - \epsilon, s + \epsilon)$ is contained within the set $A$. This contradicts the fact that $s$ is the supremum as there exists elements $s' \in (s, s + \epsilon) \subseteq A$, where $s' > s$. Hence, an open set cannot contain its supremum.
    }}
}